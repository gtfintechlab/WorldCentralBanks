\mptext{bnm}{thirteen} Economic status, Malaysian ringgit Value Change, Household loan, Foreign nations, BNM Expectations/Actions/Assets, Money Supply, Labor, Interest rate gap between Malaysia and United States, COVID, Energy/Housing Prices, and 1997 Asian Financial Crisis.

\begin{itemize}
    \item \emph{Economic Status}: A sentence pertaining to the state of the economy, relating to unemployment and inflation
    \item \emph{Malaysian ringgit Change}: A sentence pertaining to changes such as appreciation or depreciation of value of the Malaysian ringgit on the Foreign Exchange Market
    \item \emph{Household loan}: A sentence pertaining to changes in public or private debt levels.
    \item \emph{Foreign Nations}: A sentence pertaining to trade relations between Malaysia and a foreign country. If not discussing Malaysia we label neutral.
    \item \emph{BNM Expectations/Actions/Assets}: A sentence that discusses changes in the BNM yields, bond value, reserves, or any other financial asset value.
    \item \emph{Money Supply}: A sentence that overtly discusses impact to the money supply or changes in demand.
    \item \emph{Labor}: A sentence that relates to changes in labor productivity.
    \item \emph{Interest rate gap between Malaysia and United States}: A sentence pertaining to whether the US interest rate is lower or higher compared to Malaysia's interest rate.
    \item \emph{Energy/Housing Prices}: A sentence pertaining to changes in prices of real estate, energy commodities, or energy sector as a whole.
    \item \emph{1997 Asian Financial Crisis}: A sentence pertaining to when the financial crisis caused deflationary pressures or capital outflows.
\end{itemize}

\paragraph{Examples: }
\begin{itemize}
    \item ``At the Monetary Policy Committee (MPC) meeting today, Bank Negara Malaysia decided to leave the Overnight Policy Rate (OPR) unchanged at 3.50 percent.''\\
    \textbf{Neutral}: Maintaining existing interest rates indicates no change in monetary policy.
    
    \item ``Notwithstanding the stronger aggregate demand and higher global energy prices, the domestic inflation rate, as measured by the Consumer Price Index, remained low at 1.2\% in the second quarter. The underlying inflation in the economy is expected to remain low.''\\
    \textbf{Dovish}: Low inflation rates indicate room for economic growth and an accommodative monetary policy.

    \item ``Over the course of the COVID-19 crisis, the OPR was reduced by a cumulative 125 basis points to a historic low of 1.75\% to provide support to the economy. The unprecedented conditions that necessitated such actions have since abated. With the domestic growth on a firmer footing, the MPC decided to begin reducing the degree of monetary accommodation. This will be done in a measured and gradual manner, ensuring that monetary policy remains accommodative to support a sustainable economic growth in an environment of price stability.''\\
    \textbf{Hawkish}: Reducing monetary accommodation signals a shift towards a tighter economic policy.
    
    \item ``At the current OPR level, the monetary policy stance remains supportive of the economy and is consistent with the current assessment of inflation and growth prospects.''\\
    \textbf{Hawkish}: A monetary policy stance that supports growth of the economy indicates an tighter approach.
    
    \item ``Headline inflation is projected to average between 2.2\% - 3.2\% in 2022. Given the improvement in economic activity amid lingering cost pressures, underlying inflation, as measured by core inflation, is expected to trend higher to average between 2.0\% - 3.0\% in 2022.''\\
    \textbf{Hawkish}: A rise in inflation and cost pressure indicates a tighter monetary policy stance to maintain price stability.

    \item ``The MPC will continue to carefully evaluate the global and domestic economic and financial developments and their implications on the overall outlook for inflation and growth of the Malaysian economy.''\\
    \textbf{Neutral}: There is no mention of a policy shift since the MPC is solely observing the situation.
\end{itemize}

\begin{longtable}{p{0.118\textwidth}p{0.183\textwidth}p{0.183\textwidth}p{0.183\textwidth}p{0.183\textwidth}}
\caption{\mptitle{Bank Negara Malaysia}} \label{tb:bnm_mp_stance_guide} \\
\toprule
\textbf{Category} & \textbf{Dovish} & \textbf{Hawkish} & \textbf{Neutral} & \textbf{Irrelevant} \\
\midrule
\endfirsthead

\toprule
\textbf{Category} & \textbf{Dovish} & \textbf{Hawkish} & \textbf{Neutral} & \textbf{Irrelevant} \\
\midrule
\endhead

\textbf{Economic Status} & When inflation drops below BNM's 1\% target, unemployment increases, or economic growth is projected as low. & When inflation exceeds BNM's 3\% target, unemployment decreases, economic growth is projected as high, economic output surpasses potential supply, or economic slack declines. & When unemployment rate or growth is unchanged, maintained, or sustained. & Sentence is not relevant to monetary policy. \\
\midrule
\textbf{Malaysian Ringgit Value Change} & When the ringgit appreciates beyond BNM targets and inflation is low (below 1.5\%), signaling a need to boost growth. & When the ringgit depreciates beyond BNM targets and inflation is high (above 3\%), indicating a need for economic cooling. & When inflation is between 1.5\% and 3\%, and the ringgit's value changes at a healthy rate. & Sentence is not relevant to monetary policy. \\
\midrule
\textbf{Household Loan} & When public or private debt levels rise, leading to stricter fiscal policies to manage borrowing (household debt-to-GDP ratio exceeds 84\%). & When public or private debt levels decrease, allowing for more borrowing flexibility and fiscal stimulus (household debt-to-GDP ratio drops below 84\%). & N/A & Sentence is not relevant to monetary policy. \\
\midrule
\textbf{Foreign Nations} & When Malaysia's trade deficit increases with the U.S., Singapore, or Japan, signaling more exports than imports and ringgit appreciation. & When Malaysia's trade deficit rises with China or falls with the U.S., Singapore, or Japan, indicating ringgit depreciation. & When referring to foreign nations' economic or trade policies. & Sentence is not relevant to monetary policy. \\
\midrule
\textbf{BNM Expectations, Actions, and Assets} & When BNM expects subpar inflation, predicts disinflation, observes narrowing treasury bond spreads, declining treasury yields, or reduces bank reserves. & When BNM signals tightening, raises interest rates, or increases reserves to curb inflation. & When BNM believes current economic conditions are satisfactory or will self-correct. & Sentence is not relevant to monetary policy. \\
\midrule
\textbf{Money Supply} & When money supply is low, M2 growth rate falls below 4\%, and loan demand increases. & When money supply is high, M2 growth rate exceeds 6\%, demand for goods rises, and loan demand is low. & N/A & Sentence is not relevant to monetary policy. \\
\midrule
\textbf{Labor} & When productivity increases, solid labor market, strong semiconductor, electronics, and tourism industries, growing wages and consumer spending. & When productivity decreases, tourism is down, exports decreasing, downturn in global technology sector. & N/A & Sentence is not relevant to monetary policy. \\
\midrule
\textbf{Interest rate gap between Malaysia and United States} & When US interest rate is aligned with Malaysia’s, US not interfering with Malaysia’s interest rates. & When US interest rate is higher than Malaysia, potentially pressuring Malaysia to raise interest rates. & N/A & Sentence is not relevant to monetary policy. \\
\midrule
\textbf{COVID} & When easing Covid regulation: boost tourism industry and increased productivity. & When increased Covid: regulation detrimental for tourism industry and decreased productivity. & N/A & Sentence is not relevant to monetary policy. \\
\midrule
\textbf{Energy and Housing Prices} & When oil/energy prices decrease, when house prices decrease or expected to decrease. & When oil/energy prices increase, when house prices increase or expected to increase. & N/A & Sentence is not relevant to monetary policy. \\
\midrule
\textbf{1997 Asian Financial Crisis} & When the financial crisis causes deflationary pressures, prompting the central bank to lower interest rates to stimulate demand and growth. & When the financial crisis causes capital outflows, and the central bank raises interest rates to defend the currency and attract foreign capital. & N/A & Sentence is not relevant to monetary policy. \\
\bottomrule
\end{longtable}
