\mptext{boj}{eleven} Money Supply, Economic Status, Yen Value Change, Energy and Commodity Prices, Foreign Trade, Monetary Policy Stance, Inflation Expectations, Government Bonds and JGB Yields, BoJ Asset Purchases, Labor Market Conditions, and Fiscal Policy Interactions.


\begin{itemize}
    \item \emph{Money Supply}: A sentence that discusses changes in the central bank's control over the money supply of Japan. This includes adjustments in interest rates, quantitative easing policies, and shifts in inflation tolerance, which for Japan, is currently at 2\%. Japan's current quantitative easing policy involves buying \$40 billion worth of Japanese government bonds monthly.
    \item \emph{Economic Status}: A sentence that describes the overall condition of the economy, highlighting trends in inflation, consumer spending, and economic growth. 
    \item \emph{Yen Value Change}: A sentence pertaining to fluctuations in the value of the Yen on the foreign exchange market.
    \item \emph{Energy and Commodity Prices}: A sentence addressing movements in energy and commodity markets or addressing policy that is expected to cause change in energy and commodity markets. For example, lower commodity prices in 2010s allowed the Bank of Japan to keep policies loose without inflation fears.
    \item \emph{Foreign Trade}: A sentence that focuses on foreign trade (e.g., imports and exports) and its impacts on Japan's economy. For example, if Japan has a trade deficit, money would be, on net, leaving the Japanese economy. 
    \item \emph{Monetary Policy Stance (Interest Rates and actions)}: A sentence that outlines the central bank's policy direction through interest rate decisions and asset purchase programs, signaling either a dovish or hawkish stance.
    \item \emph{Inflation Expectations}: A sentence mentions inflation expectations in Japan. Japan's current inflation target is at 2\%.
    \item \emph{Government Bonds and JGB Yields}: A sentence discussing the behavior of government bond yields, including new adjustments in government bond purchases or the yield curve.
    \item \emph{BoJ Asset Purchases (ETFs, REITs, etc.)}: A sentence related to the Bank of Japan's strategy in buying financial assets such as ETFs.
    \item \emph{Labor Market Conditions}: A sentence focused on the labor market, including indicators such as unemployment rates and wage growth.
    \item \emph{Fiscal Policy Interactions}: A sentence that examines the relationships between monetary and fiscal policies, highlighting how changes in, for example,  government spending or taxation affect monetary policy.
\end{itemize}

\paragraph{Examples: }
\begin{itemize}
    \item "Some members commented that the growth in demand for some it-related goods was slowing around the world."\\
    \textbf{Dovish}: The section in the sentence about slowing demand implies that current economic pressures are subdued, increasing the need for expansionary monetary policy.
    
    \item "With regard to the first factor, one member said that inflation expectations for fiscal 2012 and 2013 had been relatively low at around 0 percent, although medium- to long-term inflation expectations remained stable at around 1.0 percent."\\
    \textbf{Dovish}: The emphasis on very low near-term inflation expectations would allow for expansionary monetary policy without inflation fears.
    
    \item "Some members said that private consumption and fixed asset investment continued to show high growth."\\
    \textbf{Hawkish}: The noted high growth in private consumption and fixed asset investment indicate an overheating economy, which may eventually generate inflationary pressures. This would ultimately require contractionary monetary policy to offset any future inflationary pressures that may arise.
    
    \item "Stock prices in emerging economies had also dropped significantly, due mainly to the rise in inflation."\\
    \textbf{Hawkish}: The sentence indicates that inflationary pressures are affecting asset markets, requiring a tightening stance to counteract the inflation.
    
    \item "The funding conditions of European financial institutions had recently been even more stable."\\
    \textbf{Neutral}: The statement objectively reports that funding conditions are stable without implying any need for either easing or tightening monetary policy.
    
    \item "The representative from the cabinet office made the following remarks."\\
    \textbf{Irrelevant}: This sentence is purely introductory and does not provide any substantive economic or policy-related information for analysis.
\end{itemize}



\newpage

\begin{longtable}{p{0.118\textwidth}p{0.183\textwidth}p{0.183\textwidth}p{0.183\textwidth}p{0.183\textwidth}}
\caption{\mptitle{Bank of Japan}} \\
\toprule
\textbf{Category} & \textbf{Hawkish} & \textbf{Dovish} & \textbf{Neutral} & \textbf{Irrelevant} \\
\midrule
\endfirsthead

\toprule
\textbf{Category} & \textbf{Hawkish} & \textbf{Dovish} & \textbf{Neutral} & \textbf{Irrelevant} \\
\midrule
\endhead

\textbf{Money Supply} & Lower interest rates, quantitative easing via bank purchases, higher inflation tolerance. & Higher interest rates on money, cease QE, lower inflation tolerance. A 0.25\% interest rate is the highest in Japan currently. & If the interest rates are staying the same. & Sentence is not relevant to monetary policy. \\
\midrule

\textbf{Economic Status} & Liquid economy, increased money flow between businesses and consumers, increasing CPI. & More rigid economy that is shrinking from a state of inflation, less exchange of money between businesses and consumers, decreasing CPI. & If the CPI stays the same or is not affected. & Sentence is not relevant to monetary policy. \\
\midrule

\textbf{Yen Value Change} & When the Yen appreciates (Ex. Allowing the yen to appreciate to reduce import costs). & When the Yen depreciates (Ex. Allowing the yen to weaken to boost exports). & Maintaining a stable Yen exchange rate. & Sentence is not relevant to monetary policy. \\
\midrule

\textbf{Energy and Commodity Prices} & When energy and commodity prices decrease, thus allowing the BoJ to continue their easing measures. & When energy and commodity prices increase. & Stable energy and commodity prices lead the BoJ to make no significant changes. & Sentence is not relevant to monetary policy. \\
\midrule

\textbf{Foreign Trade} & An increase in trade and exports. & A decrease in trade or exports to reduce inflation. & Foreign trade and exports are maintained. & Sentence is not relevant to monetary policy. \\
\midrule

\textbf{Monetary Policy Stance (Interest Rates and actions)} & Expanding asset purchases or lowering interest rates (Ex. Lowering interest rates and purchasing assets to support the economy, such actions took place during COVID-19). & Reducing asset purchases or signaling future interest rate increases (Ex. BoJ reducing its target for JGB purchases or signaling a future interest rate increase to curb inflation). & Keeping monetary policy unchanged, no signs of making interest rate changes or modifying practices of purchasing assets. & Sentence is not relevant to monetary policy. \\
\midrule

\textbf{Inflation Expectations} & High or increasing inflation. & Low or decreasing inflation (target 2\%). & If inflation is unchanged. & Sentence is not relevant to monetary policy. \\
\midrule

\textbf{Government Bonds and JGB Yields} & When BoJ increases bond purchases or targets a 10-year JGB yield near 0\% and indicates readiness to lower the yield further. & When BoJ allows a rise in 10-year JGB yields above 1\% by reducing bond purchases or adjusting yield curve control target further. & 10-year JGB yields remain within the target range (currently 0\% ± 1\%) – no significant change. & Sentence is not relevant to monetary policy. \\
\midrule

\textbf{BoJ Asset Purchases (ETFs, REITs, etc.)} & When BoJ increases asset purchases, injecting liquidity to the market. & When BoJ reduces asset purchases. & No change in the BoJs asset purchase strategy. & Sentence is not relevant to monetary policy. \\
\midrule

\textbf{Labor Market Conditions} & When unemployment rate is high or is increasing or wage growth is low or decreasing, indicating a need for accommodative policy. & When unemployment rate is low or is decreasing, or wage inflation is high or is increasing. & When unemployment rate remains between 2.5\% - 3.5\% with annual wage growth around 1.5\% - 2.5\%. & Sentence is not relevant to monetary policy. \\
\midrule

\textbf{Fiscal Policy Interactions} & Correlated with expansionary fiscal policies (Ex. BoJ gave government stimulus during crisis situations such as 2008 economic crisis and COVID-19 pandemic). & Correlated with contractionary fiscal policies (tax increases, reduced spending). & No connected changes with fiscal policy. & Sentence is not relevant to monetary policy. \\
\bottomrule
\label{tab:boj_mp_stance_guide}
\end{longtable}