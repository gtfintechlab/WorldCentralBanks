\mptext{bcrp}{twelve} Economic Status, Sol Value Change, Housing Debt/Loan, Foreign Nations, BCRP Expectations/Actions/Assets, Money Supply, Keywords/Phrases, Dollar Value, FX Market, COVID, Energy/Housing Prices, and Weather.

\begin{itemize}
    \item \emph{Economic Status}: A sentence pertaining to the state of the economy, relating to unemployment and inflation.
    \item \emph{Sol Value Change}: A sentence pertaining to changes such as appreciation or depreciation of value of the Peruvian sol on the Foreign Exchange Market
    \item \emph{Housing Debt/Loan}:  A sentence that relates to changes in debt in households.
    \item \emph{Foreign Nations}: A sentence pertaining to trade relations between Peru and a foreign country. If not discussing Peru we label neutral. 
    \item \emph{BCRP Expectations/Actions/Assets}: A sentence that discusses changes in CRBP yields, bond value, reserves, or any other financial asset value. 
    \item \emph{Money Supply}:  A sentence that overtly discusses impact to the money supply or changes in demand. 
    \item \emph{Keyword/Phrases}: A sentence that contains key word or phrase that would classify it squarely into one of the three label classes, based upon its frequent usage and meaning among particular label classes
    \item \emph{Dollar value}:  A sentence about fluctuation of the dollar’s value which can influence monetary policy in a dovish or hawkish direction.
    \item \emph{FX Market}: A sentence about stability’s changes in the foreign exchange market
    \item \emph{COVID}:  A sentence indicating whether the government is tightening or easing COVID policy
    \item \emph{Energy/Housing prices}: A sentence pertaining to changes in prices of real estate, energy commodities, or energy sector as a whole. 
    \item \emph{Weather}: A sentence pertaining to how drastic changes to weather affect the economy.
\end{itemize}

\paragraph{Examples: }
\begin{itemize}
    \item ``In light of the narrowing interest rate gap between Peru and the United States, some members suggested that the Central Reserve Bank of Peru may consider keeping interest rates low to support domestic economic recovery and encourage borrowing.''\\
    \textbf{Dovish}: This statement supports economic recovery via lowering interest rates, thus, reflecting a dovish stance.

    \item ``This measure is mainly preventive in a context of strong dynamism of domestic demand, a situation in which withdrawing monetary stimulus is advisable in order to maintain inflation within the target range.''\\
    \textbf{Hawkish}: This sentence suggests 'withdrawing monetary stimulus', which implies tightening monetary policy. 

    \item ``Therefore, between May 2022 and March 2024, the minimum legal reserve requirement rate in local currency was kept at 6 percent in the context of a progressive reduction of the monetary easing linked to the COVID-19 epidemic. This rate was lowered to 5.5 percent in April of this year in order to increase the amount of money that may be lent out and to supplement monetary easing.''\\
    \textbf{Dovish}: This statement indicates a dovish stance as it suggests monetary easing and  lowering of the legal reserve requirement rate to increase the money available for lending. 

     \item ``This higher level is explained by the new methodology that is being used to calculate this indicator since September 29, and now includes bonds involving a longer maturity in the basket''\\
    \textbf{Neutral}: This statement is purely informational, describing the change in how an indicator is calculated without any discussion of policy implications.
     \\

    \item ``The board will approve the monetary program for the next month on its session of November 5.''\\
    \textbf{Irrelevant}: This statement is purely administrative, providing details about a scheduled board session without indicating any monetary policy stance.
\end{itemize}


\newpage

\begin{longtable}{p{0.118\textwidth}p{0.183\textwidth}p{0.183\textwidth}p{0.183\textwidth}p{0.183\textwidth}}
\caption{\mptitle{Central Reserve Bank of Peru}} \label{tb:bcrp_mp_stance_guide} \\
\toprule
\textbf{Category} & \textbf{Hawkish} & \textbf{Dovish} & \textbf{Neutral} & \textbf{Irrelevant} \\
\midrule
\endfirsthead

\toprule
\textbf{Category} & \textbf{Hawkish} & \textbf{Dovish} & \textbf{Neutral} & \textbf{Irrelevant} \\
\midrule
\endhead
\textbf{Economic Status} & When inflation decreases, when unemployment increases, or when economic growth is projected as low. & When inflation increases, unemployment decreases, or economic growth is projected as high. & When unemployment rate or growth is unchanged, maintained, or sustained. & Sentence is not relevant to monetary policy. \\ 
\midrule
\textbf{Sol Value Change} & When Sol value appreciates. & When Sol value depreciates. & N/A & Sentence is not relevant to monetary policy. \\
\midrule
\textbf{Housing Debt/Loan} & When public or private debt levels decrease, signaling more flexibility for borrowing and fiscal stimulus. & When public or private debt levels increase, prompting strict fiscal policy to limit borrowing and manage rising debt. & N/A & Sentence is not relevant to monetary policy. \\
\midrule
\textbf{Foreign Nations} & When the Peruvian trade deficit decreases. & When the Peruvian trade deficit increases. & When relating to a foreign nation’s economic or trade policy. & Sentence is not relevant to monetary policy. \\
\midrule
\textbf{BCRP Expectations, Actions, and Assets} & When BCRP expects subpar inflation, disinflation, narrowing spreads of treasury bonds, decreases in treasury security yields, or reductions in bank reserves. & When BCRP signals the need for tightening, raising interest rates, or increasing reserves to curb inflation. & N/A & Sentence is not relevant to monetary policy. \\
\midrule
\textbf{Money Supply} & When the money supply is low, M2 increases, or demand for loans increases. & When the money supply is high, demand for goods increases, or demand for loans decreases. & N/A & Sentence is not relevant to monetary policy. \\
\midrule
\textbf{Keywords and Phrases} & When the stance is "accommodative," indicating a focus on "maximum employment" and "price stability." & Indicating a focus on "price stability" and "sustained growth." & Use of phrases such as “mixed,” “moderate,” or “reaffirmed.” & Sentence is not relevant to monetary policy. \\
\midrule
\textbf{Dollar Value} & When USD is moderate or weakening. & When USD strengthens, indicating potential depreciation of Sol. & N/A & Sentence is not relevant to monetary policy. \\
\midrule
\textbf{FX Market} & Stable FX market, leading to less government involvement. & Unstable FX market, leading to more government involvement. & N/A & Sentence is not relevant to monetary policy. \\
\midrule
\textbf{COVID} & Easing COVID regulations boosts the tourism industry and increases productivity. & Increased COVID regulations harm the tourism industry and decrease productivity. & N/A & Sentence is not relevant to monetary policy. \\
\midrule
\textbf{Energy and Housing Prices} & When oil/energy prices decrease or house prices decrease or are expected to decrease. & When oil/energy prices increase or house prices increase or are expected to increase. & N/A & Sentence is not relevant to monetary policy. \\
\midrule
\textbf{Weather} & Favorable weather conditions benefit agriculture, fishing, and manufacturing sectors. & Unfavorable weather conditions, such as droughts, harm agriculture, fishing, and manufacturing sectors. & N/A & Sentence is not relevant to monetary policy. \\
\bottomrule
\end{longtable}

