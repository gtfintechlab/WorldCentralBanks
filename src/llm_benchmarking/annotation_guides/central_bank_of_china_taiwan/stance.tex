\mptext{cbc}{fourteen} Interest Rates, Monetary Policy Stance, Inflation Control, Economic Growth, Exchange Rate Stability, Historical Context, GDP Growth Forecasts, Inflation Expectations, Export, Manufacturing Activity, Consumer Demand, Commodity Prices, Credit Growth, and Exchange Rate Movements. 

\begin{itemize}
    \item \emph{Interest Rates}: A sentence pertaining to the central bank's set interest rates that influence borrowing costs and overall economic liquidity.
    \item \emph{Monetary Policy Stance}: A sentence describing the direction and approach of the central bank's policy measures, including tightening or easing actions.
    \item \emph{Inflation Control}: A sentence relating to measures aimed at keeping consumer price inflation within target ranges to maintain price stability.
    \item \emph{Economic Growth}: A sentence addressing policies or indicators that reflect the growth of Taiwan's economy.
    \item \emph{Exchange Rate Stability}: A sentence related to the actions taken to maintain a stable value of the currency.
    \item \emph{Historical Context}: A sentence providing reference to past monetary policy actions to historical situations.
    \item \emph{GDP Growth Forecasts}: A sentence discussing projected changes in the Gross Domestic Product, which is often used as an indicator of economic performance.
    \item \emph{Inflation Expectations}: A sentence discussing inflation expectations, including actions or policy that would affect inflation expectations.
    \item \emph{Export}: A sentence pertaining to the performance of a country’s exports and its implications on the economy.
    \item \emph{Manufacturing Activity}: A sentence discussing manufacturing production, including metrics such as manufacturing output.
    \item \emph{Consumer Demand}: A sentence describing the spending of households and its influence on the broader economy.
    \item \emph{Commodity Prices}: A sentence relating to the fluctuations in prices of commodities such as raw materials and energy.
    \item \emph{Credit Growth}: A sentence discussing the rate at which lending is fluctuating, indicating potential shifts in financial stability.
    \item \emph{Exchange Rate Movements}: A sentence addressing changes in the currency exchange rate.
\end{itemize}

\paragraph{Examples: }
\begin{itemize}
    \item ``CBC focuses on price stability under 2\%.''\\
    \textbf{Hawkish}: The sentence signals that the central bank is committed to strict inflation control, implying a hawkish stance.
    
    \item ``During the Asian Financial Crisis, Taiwan protected its currency.''\\
    \textbf{Hawkish}: The sentence highlights a decisive response in a crisis, implying a tight monetary approach.
    
    \item ``Forecast CPI at 1.89\% supports growth.''\\
    \textbf{Dovish}: The sentence suggests a relatively low CPI, indicating that there is leeway for expansionary monetary policy.
    
    \item ``Weak orders for electronics and machinery may warrant supportive monetary steps.''\\
    \textbf{Dovish}: The sentence indicates that easing measures could be necessary to bolster export demand.
    
    \item ``CBC maintained rates at 2.00\% to ensure stability.''\\
    \textbf{Neutral}: The sentence provides a factual statement about unchanged interest rates without directional bias.

    \item ``The bank successively met with a total of 34 domestic banks between august 12 and august 21.''\\
    \textbf{Irrelevant}: The sentence describes a series of routine meetings with domestic banks and does not provide any indication of monetary stance.
\end{itemize}

\newpage

\begin{longtable}{p{0.118\textwidth}p{0.183\textwidth}p{0.183\textwidth}p{0.183\textwidth}p{0.183\textwidth}}
\caption{\mptitle{Central Bank of China}} \label{tb:cbc_mp_stance_guide}\\
\toprule
\textbf{Category} & \textbf{Hawkish} & \textbf{Dovish} & \textbf{Neutral} & \textbf{Irrelevant} \\
\midrule
\endfirsthead

\toprule
\textbf{Category} & \textbf{Hawkish} & \textbf{Dovish} & \textbf{Neutral} & \textbf{Irrelevant} \\
\midrule
\endhead

\textbf{Interest Rates} & High rates to control inflation and reduce liquidity. & Lower rates to encourage borrowing and spending. & Rates remain unchanged. & Sentence is not relevant to monetary policy. \\ 
\midrule
\textbf{Monetary Policy Stance} & Tightening measures such as reduced asset purchases or clear signals of future rate hikes. & Accommodative measures, including rate cuts, quantitative easing, or liquidity injections. & No significant change in policy stance or rates, indicating a wait-and-see approach. & Sentence is not relevant to monetary policy. \\ 
\midrule
\textbf{Inflation Control} & Strong measures to keep CPI well below 2\%. & Tolerance for higher inflation to support economic growth. & Inflation projections stable or within comfortable bounds. & Sentence is not relevant to monetary policy. \\ 
\midrule
\textbf{Economic Growth} & Policies to prevent overheating. & Measures to stimulate growth (e.g., rate cuts). & Balanced approach to growth. & Sentence is not relevant to monetary policy. \\ 
\midrule
\textbf{Exchange Rate Stability} & Aggressive forex interventions to stabilize or strengthen the currency. & Allowing slight depreciation to boost exports; looser approach to FX. & Stable exchange rates are maintained. & Sentence is not relevant to monetary policy. \\ 
\midrule
\textbf{Historical Context} & Tightening responses in past crises. & Accommodative responses in past downturns. & Mentions of periods without major policy shifts. & Sentence is not relevant to monetary policy. \\ 
\midrule
\textbf{GDP Growth Forecasts} & Focus on curbing excessive expansion to mitigate inflation. & Policies that drive higher growth through domestic demand or exports. & Stable projections indicating moderate, sustainable growth. & Sentence is not relevant to monetary policy. \\ 
\midrule
\textbf{Inflation Expectations} & Very strict control near 2\% or lower to avoid price instability. & Willingness to let inflation rise slightly if beneficial for growth. & Projections remain moderate and steady. & Sentence is not relevant to monetary policy. \\ 
\midrule
\textbf{Export} & Robust export growth, risking overheating and trade imbalances. & Declining export demand, suggesting easing measures to support exporters. & Stable export levels with sustainable growth patterns. & Sentence is not relevant to monetary policy. \\ 
\midrule
\textbf{Manufact-uring Activity} & Rapid output potentially causing input shortages or inflation pressures. & Decline in factory output due to weak global demand or disruptions. & Balanced output growth indicating stable demand in key markets. & Sentence is not relevant to monetary policy. \\ 
\midrule
\textbf{Consumer Demand} & Rising domestic consumption pushing inflation beyond targets. & Weak household spending needing policy stimulus. & Stable consumer demand, supporting GDP without driving up inflation. & Sentence is not relevant to monetary policy. \\ 
\midrule
\textbf{Commodity Prices} & Rising global commodity costs (energy, raw materials) pressuring margins. & Falling import prices easing inflation and supporting growth. & Stable commodity prices reducing uncertainty in production and exports. & Sentence is not relevant to monetary policy. \\ 
\midrule
\textbf{Credit Growth} & Rapid credit expansion increasing leverage and financial instability risks. & Slowing credit growth indicating tight conditions, possibly needing easing. & Moderate credit growth aligned with economic needs (e.g., SMEs). & Sentence is not relevant to monetary policy. \\ 
\midrule
\textbf{Exchange Rate Movements} & NTD appreciation, harming exports and requiring tighter interventions. & NTD depreciation, helping exports but risking imported inflation. & Stable rates ensuring balanced trade flows with minimal inflation pressure. & Sentence is not relevant to monetary policy. \\
\bottomrule
\end{longtable}