\mptext{mas}{six} Exchange rate policy, Inflation control, Employment, Trade, Foreign labor and Talent attraction, and Global Supply Chain Hub.

\begin{itemize}
    \item \emph{Exchange Rate Policy}: A sentence pertaining to changes such as the appreciation or depreciation of the value of the Singapore dollar (SGD).
    \item \emph{Inflation Control}: A sentence pertaining to adjusting the inflationary pressure by weakening or strengthening the SGD.
    \item \emph{Employment}: A sentence pertaining to changes in the employment sectors.
    \item \emph{Trade}: A sentence pertaining to trade relations between Singapore and a foreign country. If not discussing Singapore, we label irrelevant.
    \item \emph{Foreign Labor and Talent Attraction}: A sentence that discusses changes in immigration policies to increase or limit foreign talent.
    \item \emph{Global Supply Chain Hub}: A sentence that discusses Singapore's involvement in the global supply chain.
\end{itemize}

\paragraph{Examples: }
\begin{itemize}
    \item ``To combat rising inflation, the central bank may need to steepen the exchange rate band.''\\
    \textbf{Hawkish}: Increasing the exchange rate band indicates a tighter monetary policy stance to maintain price stability.
    
    \item ``A flattening of the band would provide greater flexibility during economic downturns.''\\
    \textbf{Dovish}: Flattening the exchange rate band indicates a more accommodative stance to support the economy.

    \item ``The current monetary policy remains effective without any proposed changes.''\\
    \textbf{Hawkish}: There are no changes made to the monetary policy and remains neutral.
    
    \item ``The committee discussed the impact of the weather on agricultural output, which is irrelevant to monetary policy.''\\
    \textbf{Irrelevant}: Agricultural output does not have an impact on the monetary policy.
    
    \item ``External inflationary pressures are likely to stay muted for the rest of 2003.''\\
    \textbf{Dovish}: The statement suggests that inflationary pressures are not expected to worsen, thereby creating room for expansionary monetary policy.

    \item ``MAS will be shifting the schedule of its semi-annual cycle from January/July to April/October.''\\
    \textbf{Irrelevant}: This sentence relates to an operational or scheduling change and does not mention monetary policy decisions or stances.
    

\end{itemize}

\newpage

\begin{longtable}{p{0.118\textwidth}p{0.183\textwidth}p{0.183\textwidth}p{0.183\textwidth}p{0.183\textwidth}}
\caption{\mptitle{Monetary Authority of Singapore}} \label{tb:mas_mp_stance_guide} \\
\toprule
\textbf{Category} & \textbf{Hawkish} & \textbf{Dovish} & \textbf{Neutral} & \textbf{Irrelevant} \\
\midrule
\endfirsthead

\toprule
\textbf{Category} & \textbf{Hawkish} & \textbf{Dovish} & \textbf{Neutral} & \textbf{Irrelevant} \\
\midrule
\endhead
\textbf{Exchange Rate Policy} & In favor of or advocating for the appreciation of the Singapore dollar (SGD). & In favor of or advocating for the depreciation or weakening of the Singapore dollar (SGD). & Identifying economic trends but not suggestive of action to the currency. & Sentence is not relevant to monetary policy. \\
\midrule
\textbf{Inflation Control} & Favoring a stronger SGD to reduce inflationary pressures. & Tending towards weakening or maintaining the SGD due to the reduction of inflationary pressures. & Discusses inflation as a whole and mentions statistics but don’t take a stance on it. & Sentence is not relevant to monetary policy. \\
\midrule
\textbf{Employment} & Lack of employment due to the strengthening of the SGD due to other concerns. & Favoring a weaker SGD to boost export-driven employment sectors. & Discusses employment metrics without suggesting action either way. & Sentence is not relevant to monetary policy. \\
\midrule
\textbf{Trade} & Favoring stricter trade policies and protective measures to guard domestically produced items against international competition. & Favoring open markets and more relaxed policies to boost exports and secure Singapore’s role as a trade hub. & Discusses previous trade strategies or options for future opportunities. & Sentence is not relevant to monetary policy. \\
\midrule
\textbf{Foreign Labor Policy} & Sentences discuss the need for stricter immigration policies in order to keep local jobs and limit foreign labor. & Suggesting the allowing of an increase in foreign talent in the local markets to foster innovation and economic growth. & Discusses the maintaining of immigration policies and SGD or mentioning local vs. foreign labor forces. & Sentence is not relevant to monetary policy. \\
\midrule
\textbf{Global Supply Chain} & Suggesting the mitigation of global supply chain disruptions by focusing on more local sourcing. & Sentences implies the increase in involvement in global supply chain in order to enhance Singapore’s position as a hub and to invite investment. & Maintaining Signapore’s policy and involvement in the global supply chain due to the lack of concerns of disruptions or happiness with Singapore’s place as the hub. & Sentence is not relevant to monetary policy. \\
\bottomrule
\end{longtable}