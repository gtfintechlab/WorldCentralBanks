\mptext{nbp}{13} Interest Rates, Inflation Rate, GDP Growth, Unemployment Rate, Required Reserve Ratio, Money Supply, Foreign Exchange Reserves, Private Consumption, Prices, Lending, Domestic Manufacturing, Gold Holdings, Net Exports.

\begin{itemize}
    \item \emph{Interest Rates}: A sentence pertaining to various interest rates used by the NBP. This includes, but is not limited to, the reference rate, Lombard rate, deposit rate, discount rate, and inflation rate. 
    \item \emph{Inflation Rate}: A sentence pertaining to the inflation rate (the change in the consumer price index).
    \item \emph{GDP Growth}: A sentence pertaining to the GDP growth of Poland.
    \item \emph{Unemployment Rate}: A sentence pertaining to the unemployment rate of Poland.
    \item \emph{Required Reserve Ratio}: A sentence pertaining to the required reserve ratio, that is, the minimum percentage of reservable liabilities that depository institutions (e.g. banks) must hold at any given time.
    \item \emph{Money Supply}: A sentence that discusses the impact to money supply or changes in demand.
    \item \emph{Foreign Exchange Reserves}: A sentence pertaining to foreign assets controlled by the NBP.
    \item \emph{Private Consumption}: A sentence pertaining to private consumption in Poland.
    \item \emph{Prices}: A sentence pertaining to a change in prices of goods and services in Poland.
    \item \emph{Lending}: A sentence that discusses changes in lending behaviors of actors such as corporations.
    \item \emph{Domestic Manufacturing}: A sentence discusses the growth or output of the manufacturing industry in Poland.
    \item \emph{Gold Holdings}: A sentence pertaining to the gold holdings of the NBP.
    \item \emph{Net Exports}: A sentence discussion the net exports of Poland.
\end{itemize}

\paragraph{Examples: }
\begin{itemize}
    \item ``The Council judged that the current level of the NBP interest rates was conducive to meeting the NBP inflation target in the medium term''\\
    \textbf{Neutral}: Interested rates are expected to meeting the inflation target, signaling that no change should be made.
    
    \item ``Some Council members pointed to a continually high rate of wage growth and to a very fast drop in unemployment, facilitating a further strong rise in wages.''\\
    \textbf{Hawkish}: High wage growth and low unemployment indicate wage-driven inflation, requiring tighter monetary policy to combat.

    \item ``Some Council members stressed that in the event of significant price growth that would jeopardize meeting the inflation target in the medium term, it might be justified to consider an increase in the NBP interest rates in the coming quarters.''\\
    \textbf{Hawkish}: The emphasis on potential interest rates increases to address price growth signals commitment to tighten monetary policy amid inflation risks.
    
    \item ``These meeting participants also assessed that the acceleration of interest rate increases might not reduce inflation expectations of households, which are highly adaptive in nature.''\\
    \textbf{Dovish}: Participants questions the effectiveness of rate hikes in lowering inflation expectations, suggesting a looser monetary policy stance.

    \item ``In effect, it was judged that the annual GDP growth in 2022 Q4 slowed down markedly for another consecutive quarter.''\\
    \textbf{Dovish}: Slowing GDP growth indicates a need for easing to stimulate economic momentum.

    \item ``The discussion at the meeting focused on the outlook for economic growth abroad and in Poland, fiscal policy, zloty exchange rate developments and the situation in the credit market and the banking sector.''\\
    \textbf{Irrelevant}: The sentence discusses the topics of the meeting, indicating no monetary policy stance.
\end{itemize}

\newpage


\begin{longtable}{p{0.118\textwidth}p{0.183\textwidth}p{0.183\textwidth}p{0.183\textwidth}p{0.183\textwidth}}
\caption{\mptitle{Narodowy Bank Polski}} \label{tb:nbp_mp_stance_guide} \\
\toprule
\textbf{Category} & \textbf{Hawkish} & \textbf{Dovish} & \textbf{Neutral} & \textbf{Irrelevant} \\
\midrule
\endfirsthead

\toprule
\textbf{Category} & \textbf{Hawkish} & \textbf{Dovish} & \textbf{Neutral} & \textbf{Irrelevant} \\
\midrule
\endhead
\textbf{Interest Rates} & When interest rates decrease. & When interest rates increase. & When interest rates are unchanged, maintained, or sustained. & Sentence is not relevant to monetary policy. \\
\midrule
\textbf{Inflation Rate} & When inflation is below target. & When inflation is above target. & When inflation is at target. & Sentence is not relevant to monetary policy. \\
\midrule
\textbf{GDP Growth} & Slowing GDP growth. & When GDP growth is at a sustained increase above potential. & Stable GDP growth. & Sentence is not relevant to monetary policy. \\
\midrule
\textbf{Unemploy-ment Rate} & Increase in unemployment rates. & Decrease in unemployment rates. & Stable unemployment rates. & Sentence is not relevant to monetary policy. \\
\midrule
\textbf{Required Reserve Ratio} & Decrease in required reserve ratio. & Increase in required reserve ratio. & Maintaining the required reserve ratio. & Sentence is not relevant to monetary policy. \\
\midrule
\textbf{Money Supply} & Low money supply or increased demand for loans. & Money supply is high, increased demand for goods, low demand for loans. & Money supply is stable. & Sentence is not relevant to monetary policy. \\
\midrule
\textbf{Foreign Exchange Reserves} & Increasing reserves, indicating boosting market confidence. & Decreasing reserves. & Maintained reserves. & Sentence is not relevant to monetary policy. \\
\midrule
\textbf{Private Consumption} & Decrease in consumption due to weak consumer sentiment. & Increase in consumption due to rising wages or government spending. & Sustained private consumption. & Sentence is not relevant to monetary policy. \\
\midrule
\textbf{Prices} & Falling prices indicating reduced inflation. & Rising prices indicating increased inflation. & Maintaining existing price levels. & Sentence is not relevant to monetary policy. \\
\midrule
\textbf{Corporate Lending} & Decline in lending, low investment. & Rapid increase in corporate borrowing. & Maintaining existing lending levels. & Sentence is not relevant to monetary policy. \\
\midrule
\textbf{Domestic Manufacturing} & Weak domestic manufacturing output or growth. & Significant domestic manufacturing output or growth. & Stable domestic manufacturing output and growth. & Sentence is not relevant to monetary policy. \\
\midrule
\textbf{Gold Holdings} & Falling gold holdings indicating flexibility in monetary policy. & Rising gold holdings indicating tighter monetary policy. & Maintained gold holdings. & Sentence is not relevant to monetary policy. \\
\midrule
\textbf{Net Exports} & Low net exports indicative of low demand or uncompetitive industries. & High net exports, indicating higher risk of inflation. & Maintained net export levels. & Sentence is not relevant to monetary policy. \\
\bottomrule
\end{longtable}