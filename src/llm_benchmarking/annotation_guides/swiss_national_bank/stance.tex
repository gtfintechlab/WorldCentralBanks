\mptext{snb}{Nine} Interest Rates, Inflation Rate, GDP Growth Forecast, Export Performance, Consumer Demand, Employment Levels, Exchange Rate Movements, Banking Industry, and Tourism. 
\begin{itemize}
    \item \textit{Interest Rates: } A sentence pertaining to changes in interest rates or the Swiss Average Rate Overnight (SARON).
    \item \textit{Inflation Rate: }A sentence pertaining to the inflation rate, consumer price index, or some other inflation-related metric  or references price stability.
    \item \textit{GDP Growth Forecast: }A sentence pertaining to future economic growth expectations.
    \item \textit{Export Performance: }A sentence pertaining to the export level 
    \item \textit{Consumer Demand: }A sentence pertaining to consumer demand and consumer spending level  
    \item \textit{Employment Levels: }A sentence pertaining to employment
    \item \textit{Exchange Rate Movements: }A sentence pertaining to the value of the Swiss Franc.
    \item \textit{Banking Industry: }A sentence pertaining to Switzerland's banking industry.
    \item \textit{Tourism: }A sentence pertaining to the tourism level in Switzerland.
\end{itemize}


\textbf{Examples:}
\begin{itemize}
    \item "The economy has remained relatively robust despite the deterioration in the world economy, but the slowdown is likely to continue over the months to come."\newline\textbf{Dovish: }The economic is projected to slowdown so the SNB will be inclined to promote spending.
    \item "Since the indicators for the demand for labour are still at a high level, the rise in employment is likely to continue in the first part of 2008."\newline\textbf{Hawkish: }The SNB expects employment to increase causing more money to enter into the economy.
    \item "Inflationary pressure from abroad will remain weak."\newline\textbf{Neutral: }This statement does not directly comment on the state of the Swiss economy. 
    \item "Through to the second quarter of 2007, this curve is a little higher than for June."\newline
    \textbf{Irrelevant: }It is unclear what this statement is referring to. 
    
\end{itemize}


\newpage


\begin{longtable}{p{0.118\textwidth}p{0.183\textwidth}p{0.183\textwidth}p{0.183\textwidth}p{0.183\textwidth}}
\caption{\mptitle{Swiss National Bank}} \label{tb:snb_mp_stance_guide}\\ 
\toprule
\textbf{Category} & \textbf{Hawkish} & \textbf{Dovish} & \textbf{Neutral} & \textbf{Irrelevant} \\
\midrule
\endfirsthead

\toprule
\textbf{Category} & \textbf{Hawkish} & \textbf{Dovish} & \textbf{Neutral} & \textbf{Irrelevant} \\
\midrule
\endhead
\textbf{Interest Rates} & When interest rate or SARON rate is high. & When interest rate or SARON rate is low. & When interest rate or SARON rate is stable. & Sentence is not relevant to monetary policy. \\
\midrule
\textbf{Inflation Rate} & When inflation more than the target of 2\%. & When inflation is below the desired level. & When is at the target of 0-2\%. & Sentence is not relevant to monetary policy. \\
\midrule
\textbf{GDP Growth Forecast} & When GDP is increasing. & When GDP is decreasing. & When GDP growth rate is stable. & Sentence is not relevant to monetary policy. \\
\midrule
\textbf{Export Performance} & When exports are performing better than expected and earnings are rising. & When exports are declining and earnings are falling. & When exports are as expected and the rate is sustainable. & Sentence is not relevant to monetary policy. \\
\midrule
\textbf{Consumer Demand} & When consumer demand is higher than expected and inflation is increasing. & When consumer demand is weak and spending is low a. & When the economy is growing at a sustainable rate in line with expectations. & Sentence is not relevant to monetary policy. \\
\midrule
\textbf{Employment Levels} & When unemployment is very low. & When unemployment is high. & Unemployment is at a sustainable level and within projections. & Sentence is not relevant to monetary policy. \\
\midrule
\textbf{Exchange Rate Movements} & When Franc is depreciating. & When currency is appreciating. & When currency value is stable. & Sentence is not relevant to monetary policy. \\
\midrule
\textbf{Banking Industry} & When the banking industry is growing too quickly. & When the global economy is weak and financial activity is low, harming the bank-dependent Swiss economy. & When the banking industry is stable. & Sentence is not relevant to monetary policy. \\
\midrule
\textbf{Tourism} & When the tourism season is very active. & When tourism levels are weak. & When tourism levels are as expected. & Sentence is not relevant to monetary policy. \\
\bottomrule
\end{longtable}
