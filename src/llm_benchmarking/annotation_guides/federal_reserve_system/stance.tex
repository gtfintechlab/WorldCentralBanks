\mptext{fomc}{eight} Economic Status, Dollar Value Change, Energy/House Prices, Foreign Nations, Fed Expectations/Actions/Assets, Money Supply, Key Words/Phrases, corporate bond, and Labor. 

\begin{itemize}
    \item \emph{Economic Status}: A sentence pertaining to the state of the economy, relating to unemployment and inflation.
    \item \emph{Dollar Value Change}: A sentence pertaining to changes such as appreciation or depreciation of value of the United States Dollar on the Foreign Exchange Market.
    \item \emph{Energy/House Prices}: A sentence pertaining to changes in prices of real estate, energy commodities, or energy sector as a whole.
    \item \emph{Foreign Nations}: A sentence pertaining to trade relations between the United States and a foreign country. If not discussing the United States we label neutral.
    \item \emph{Fed Expectations/Actions/Assets}: A sentence that discusses changes in the Fed yields, bond value, reserves, or any other financial asset value.
    \item \emph{Money Supply}: A sentence that overtly discusses impact to the money supply or changes in demand.
    \item \emph{Key Words/Phrases}: A sentence that contains key word or phrase that would classify it squarely into one of the three label classes, based upon its frequent usage and meaning among particular label classes.
    \item \emph{Corporate Bond}: A sentence that relates to corporate bond issuance.
    \item \emph{Labor}: A sentence that relates to changes in labor productivity.
\end{itemize}

\paragraph{Examples: }
\begin{itemize}
    \item ``The Committee then turned to a discussion of the economic and financial outlook, the ranges for the growth of money and debt in 1996, and the implementation of monetary policy over the intermeeting period ahead.''\\
    \textbf{Neutral}: Economic status is only mentioned without diving deeper into increases or decreases in inflation and/or unemployment rate.
    
    \item ``To support the Committee’s decision to raise the target range for the federal funds rate, the Board of Governors voted unanimously to raise the interest rates on required and excess reserve balances to 2.''\\
    \textbf{Hawkish}: The Fed expects higher inflation so they are likely to increase interest rates.
    
    \item ``Labor productivity has continued to rise over recent months, supporting stronger economic growth and reducing inflationary pressures.''\\
    \textbf{Dovish}: The increase in labor productivity can ease inflation concerns without requiring tighter monetary policy.
    
    \item ``The U.S. trade deficit has widened significantly due to increased imports from foreign nations, contributing to upward pressure on inflation.''\\
    \textbf{Hawkish}: A growing trade deficit can fuel inflation, prompting the Fed to consider tightening monetary policy to control price stability.
    
    \item ``Real GDP was anticipated to increase at a rate noticeably below its potential in 2008.''\\
    \textbf{Dovish}: Expected GDP below potential indicates that economy is slowing down, requiring expansionary monetary policy to prevent any drawbacks.


    \item ``It was agreed that the next meeting of the committee would be held on Tuesday-Wednesday, April 26-27, 2016.''\\
    \textbf{Irrelevant}: This sentence solely provides scheduling information and does not offer any insights into a monetary policy stance.
\end{itemize}

\newpage

\begin{longtable}{p{0.118\textwidth}p{0.183\textwidth}p{0.183\textwidth}p{0.183\textwidth}p{0.183\textwidth}}
\caption{\mptitle{The Federal Open Market Committee}} \\
\toprule
\textbf{Category} & \textbf{Hawkish} & \textbf{Dovish} & \textbf{Neutral} & \textbf{Irrelevant} \\
\midrule
\endfirsthead

\toprule
\textbf{Category} & \textbf{Hawkish} & \textbf{Dovish} & \textbf{Neutral} & \textbf{Irrelevant} \\
\midrule
\endhead

\textbf{Economic Status} & When inflation increases, when unemployment decreases, when economic growth is projected high, when economic output is higher than potential supply/actual output, when economic slack falls. & When inflation decreases, when unemployment increases, when economic growth is projected as low. & When unemployment rate or growth is unchanged, maintained, or sustained. & Sentence is not relevant to monetary policy. \\
\midrule
\textbf{Dollar Value Change} & When the dollar depreciates. & When the dollar appreciates. & N/A & Sentence is not relevant to monetary policy. \\
\midrule
\textbf{Energy/House Prices} & When oil/energy prices increase, when house prices increase. & When oil/energy prices decrease, when house prices decrease. & N/A & Sentence is not relevant to monetary policy. \\
\midrule
\textbf{Foreign Nations} & When the US trade deficit increases. & When the US trade deficit decreases. & When relating to a foreign nation's economic or trade policy. & Sentence is not relevant to monetary policy. \\
\midrule
\textbf{Fed Expectations, Actions, and Assets} & Fed expects high inflation, widening spreads of treasury bonds, increase in treasury security yields, increase in TIPS value, increase bank reserves. & Fed expects subpar inflation, Fed expecting disinflation, narrowing spreads of treasury bonds, decreases in treasury security yields, and reduction of bank reserves. & N/A & Sentence is not relevant to monetary policy. \\
\midrule
\textbf{Money Supply} & Money supply is high, increased demand for goods, low demand for loans. & Money supply is low, M2 increases, increased demand for loans. & N/A & Sentence is not relevant to monetary policy. \\
\midrule
\textbf{Key Words and Phrases} & Indicating a focus on “price stability” and “sustained growth.” & When the stance is ``accommodative,'' indicating a focus on “maximum employment” and “price stability.” & Use of phrases “mixed,” “moderate,” “reaffirmed.” & Sentence is not relevant to monetary policy. \\
\midrule
\textbf{Corporate Bond} & When issuance increases. & When issuance decreases. & N/A & Sentence is not relevant to monetary policy. \\
\midrule
\textbf{Labor} & When productivity or unemployment decreases. & When productivity or unemployment increases. & N/A & Sentence is not relevant to monetary policy. \\
\bottomrule
\label{tb:fomc_mp_stance_guide}
\end{longtable}