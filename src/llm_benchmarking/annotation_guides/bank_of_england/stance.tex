\mptext{boe}{eight} Economic Status, Pound Value Change, Energy/House Prices, Foreign Nations, BoE Expectations, Actions, Assets, Money Supply, Key Words, Phrases, and Labor.

\begin{itemize}
    \item \emph{Economic Status}: A sentence pertaining to the overall condition of the economy, including indicators such as inflation, unemployment, and growth projections.
    \item \emph{Pound Value Change}: A sentence pertaining to fluctuations in the value of the British Pound on the foreign exchange market, indicating whether it is appreciating or depreciating.
    \item \emph{Energy/House Prices}: A sentence pertaining to variations in the prices of energy commodities and residential properties.
    \item \emph{Foreign Nations}: A sentence pertaining to the economic impacts of trade relations with foreign countries, particularly those affecting the United Kingdom.
    \item \emph{BoE Expectations, Actions, Assets}: A sentence that discusses the Bank of England’s projections, policy actions, and changes in financial assets such as treasury bonds and bank reserves.
    \item \emph{Money Supply}: A sentence that examines changes in the overall availability of money in the economy, including shifts in loan demand and monetary liquidity.
    \item \emph{Key Words, Phrases}: A sentence that includes specific keywords or phrases which clearly indicate a particular monetary policy stance.
    \item \emph{Labor}: A sentence that relates to shifts in labor market conditions, encompassing changes in productivity, wage trends, and job openings.
\end{itemize}

\paragraph{Examples: }
\begin{itemize}
    \item ``Recent economic indicators showing rising inflation and falling unemployment have led the Bank of England to signal a tighter monetary policy stance.''\\
    \textbf{Hawkish}: The BOE is concerned that overheating growth will fuel further inflation, prompting interest rate hikes.
    
    \item ``The acceleration of house prices and energy costs is causing upward pressures on consumer demand, which may necessitate a proactive tightening of monetary policy.''\\
    \textbf{Hawkish}: Rising asset prices and energy costs are seen as signals of mounting inflationary pressures.
    
    \item ``Falling inflation paired with a rise in unemployment has convinced policymakers at the Bank of England that a more accommodative stance is needed to support the economy.''\\
    \textbf{Dovish}: Lower inflation and softer labor market conditions suggest that further easing could help sustain economic growth.
    
    \item ``Low money supply growth combined with persistently high loan demand has reinforced the BOE's outlook for maintaining an expansionary monetary policy stance.''\\
    \textbf{Dovish}: These factors indicate subdued economic activity, reducing the likelihood of immediate monetary policy tightening.

    \item ``Total output prices had been flat year on year.''\\
    \textbf{Neutral}: The sentence states that total output prices remained stable and doesn't take a direction on monetary policy.
    
    \item ``The new government had announced its intention to present an updated budget on 8 July.''\\
    \textbf{Irrelevant}: The announcement pertains to a general government schedule and does not contain a monetary policy stance.
\end{itemize}

\newpage
\begin{longtable}{p{0.118\textwidth}p{0.183\textwidth}p{0.183\textwidth}p{0.183\textwidth}p{0.183\textwidth}}
\caption{\mptitle{Bank of England}}     
\label{tb:boe_mp_stance_guide}
\\
\toprule
\textbf{Category} & \textbf{Dovish} & \textbf{Hawkish} & \textbf{Neutral} & \textbf{Irrelevant} \\
\midrule
\endfirsthead

\toprule
\textbf{Category} & \textbf{Dovish} & \textbf{Hawkish} & \textbf{Neutral} & \textbf{Irrelevant} \\
\midrule
\endhead

\textbf{Economic Status} & Inflation decreases, unemployment increases, low projected growth. & Inflation increases, unemployment decreases, high projected growth, demand outgrows supply. & When unemployment rate or growth is unchanged, maintained, or sustained. & Sentence is not relevant to monetary policy. \\
\midrule
\textbf{Pound Value Change} & Pound appreciates. & Pound depreciates. & N/A. & Sentence is not relevant to monetary policy. \\
\midrule
\textbf{Energy andHouse Prices} & Oil/energy prices decrease, house prices decrease. & Oil/energy prices increase, house prices increase. & N/A. & Sentence is not relevant to monetary policy. \\
\midrule
\textbf{Foreign Nations} & UK trade deficit decreases. & UK trade deficit increases. & Relating to foreign economic policy. & Sentence is not relevant to monetary policy. \\
\midrule
\textbf{BoE Expectations, Actions, Assets} & Expects subpar inflation, disinflation, narrowing spreads of treasury bonds, decreases in treasury security yields, and reduction of bank reserves. & Expects high inflation, widening spreads of treasury bonds, increase in treasury security yields, increase in TIPS value, increase bank reserves. & N/A. & Sentence is not relevant to monetary policy. \\
\midrule
\textbf{Money Supply} & Low money supply, slow M4 growth, increased loan demand. & High money supply, increased demand for goods, low loan demand. & N/A. & Sentence is not relevant to monetary policy. \\
\midrule
\textbf{Key Words and Phrases} & Stance is ``accommodative,'' indicating a focus on ``maximum employment'' and ``price stability.'' & Indicative a focus on ``price stability'' and ``sustained growth.'' & Usage of ``mixed,'' ``moderate,'' ``reaffirmed.'' & Sentence is not relevant to monetary policy. \\
\midrule
\textbf{Labor} & Productivity increases, unemployment decreases, wages increase, job openings increase. & Productivity decreases, unemployment increases, wages decrease, job openings decrease. & N/A. & Sentence is not relevant to monetary policy. \\
\bottomrule
\end{longtable}

