
\mptext{boi}{10} Interest Rate, Capital Market Inflation Rate, Total Assets, GDP (Gross Domestic Product), Unemployment Rate, Job Vacancy Rate, NIS to Euro (or Dollar), Inflation, Housing Market, War.

\textbf{Important Factors}
\begin{itemize}
    \item \textit{Interest Rate:} The Bank of Israel has adjusted its benchmark interest rate, influencing the cost of borrowing and impacting economic activity by either stimulating or cooling down demand.
    \item \textit{Capital Market Inflation Rate:} Inflation in Israel's financial markets has been reflected in fluctuations in asset prices, with stock and bond markets responding to macroeconomic conditions and central bank policies.
    \item \textit{Total Assets:} The Bank of Israel's balance sheet has expanded as foreign reserves increased, injecting liquidity into the financial system and affecting overall economic stability.
    \item \textit{GDP:} Israel’s GDP growth has been influenced by the central bank’s monetary policies, balancing inflation control with efforts to sustain economic expansion.
    \item \textit{Unemployment Rate:} The unemployment rate has shifted due to monetary policy measures aimed at either job creation or inflation containment, affecting labor market dynamics.
    \item \textit{Job Vacancy Rate:} Changes in the job vacancy rate indicate shifts in labor demand, reflecting the impact of central bank policies on hiring trends and wage pressures.
    \item \textit{NIS to Euro (or Dollar):} The exchange rate of the Israeli shekel against the Euro and the U.S. dollar has fluctuated, influencing import costs, export competitiveness, and inflation levels.
    \item \textit{Inflation:} The Bank of Israel has implemented monetary policies to stabilize inflation, ensuring that price levels remain within the target range for economic stability.
    \item \textit{Housing Market:} Interest rate adjustments have directly affected mortgage rates, impacting housing affordability and real estate market activity.
    \item \textit{War:} Ongoing geopolitical conflicts have introduced economic uncertainty, affecting market confidence, trade flows, and the central bank’s policy responses.
\end{itemize}


\paragraph{Examples: }
\begin{itemize}
    \item ``The Monetary Committee decided to leave the interest rate unchanged, but sees a real possibility of having to raise the interest rate in future decisions.''\\
    \textbf{Hawkish}: This sentence implies that despite the current stance, future policy may tighten, reflecting caution against inflation.
    
    \item ``The Bank of Israel increased bond purchases and foreign currency reserves.''\\
    \textbf{Hawkish}: This example indicates a proactive measure to curb potential overheating of the economy.
    
    \item ``The Committee members agreed that if necessary, the Committee will take additional steps to make monetary policy even more accommodative.''\\
    \textbf{Dovish}: The sentence suggests a readiness to ease monetary conditions in order to support economic growth.
    
    \item ``The Committee decided to sell bonds and securities to reduce the money supply.''\\
    \textbf{Dovish}: By selling bonds, the bank is expanding its balance sheet, a typical sign of an easing or accommodative policy stance.
    
    \item ``It was decided to keep the interest rate unchanged at 0.1 percent.''\\
    \textbf{Neutral}: This sentence reflects a steady policy stance with no movement towards tightening or easing.

    \item ``Other participants in the discussion are the directors of the research and market operations departments, and economists from various departments who prepare and present the material for discussion.''
    \textbf{Irrelevant}: This sentence discusses the participants in a discussion rather than focusing on monetary policy or actions that influence monetary policy decisions.
\end{itemize}

\newpage


\begin{longtable}{p{0.118\textwidth}p{0.183\textwidth}p{0.183\textwidth}p{0.183\textwidth}p{0.183\textwidth}}
\caption{\mptitle{Bank of Israel}} \label{tb:boi_mp_stance_guide} \\

\toprule
\textbf{Category} & \textbf{Dovish} & \textbf{Hawkish} & \textbf{Neutral} & \textbf{Irrelevant} \\
\midrule
\endfirsthead

\toprule
\textbf{Category} & \textbf{Dovish} & \textbf{Hawkish} & \textbf{Neutral} & \textbf{Irrelevant} \\
\midrule
\endhead

\textbf{Interest Rate} & 
When interest rates are too high, less money to spend, reducing economic growth. & 
When interest rates are too low, lots of borrowing, reduces value of money. & 
When interest rates remain the same. & 
Sentence is not relevant to monetary policy. \\
\midrule

\textbf{Capital Market Inflation Rate} & 
Likely to accept higher inflation for lower rates. & 
Aim to keep inflation low with tighter policies. & 
Stable inflation without policy shifts. & 
Sentence is not relevant to monetary policy. \\
\midrule

\textbf{Total Assets} & 
Expanding balance sheets, such as selling bonds. & 
Reducing balance sheet to protect against overheating. & 
Maintaining a stable balance sheet. & 
Sentence is not relevant to monetary policy. \\
\midrule

\textbf{GDP} & 
Shrinking GDP indicating need for stimulus. & 
Rising GDP increases inflation/employment. & 
GDP remains steady. & 
Sentence is not relevant to monetary policy. \\
\midrule

\textbf{Unemploy-ment Rate} & 
Focus on lowering unemployment. & 
Willing to tolerate higher unemployment due to inflation. & 
Balancing unemployment and inflation. & 
Sentence is not relevant to monetary policy. \\
\midrule

\textbf{Job Vacancy Rate} & 
Stimulating economy to fill vacancies. & 
High vacancies as wage inflation signal; tighter stance. & 
Monitored but not policy-shaping. & 
Sentence is not relevant to monetary policy. \\
\midrule

\textbf{NIS to Euro (or Dollar)} & 
High NIS makes exports expensive, lowering trade. & 
Higher interest rates to make NIS more attractive. & 
Inflation at target, balanced economy. & 
Sentence is not relevant to monetary policy. \\
\midrule

\textbf{Housing Market} & 
Lower interest rates make mortgages more affordable. & 
Higher rates discourage borrowing, cool prices. & 
Inflation and rates at target levels. & 
Sentence is not relevant to monetary policy. \\
\midrule

\textbf{War} & 
Lower risk perception, lower rates encouraged. & 
Higher risk perception, tighter policy. & 
War impact unchanged. & 
Sentence is not relevant to monetary policy. \\
\bottomrule
\end{longtable}

