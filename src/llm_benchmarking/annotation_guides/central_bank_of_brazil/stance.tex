\mptext{bcb}{twelve} Selic Rate, Monetary Policy, Inflation, Aggregate Demand, Food Prices, Industrial Goods Prices, Services Prices, Labor Market, GDP Growth, Exchange Rate, Trade Deficit, and the Cattle Cycle.

\begin{itemize}
    \item \emph{Selic Rate}: The benchmark interest rate set by Brazil's central bank.
    \item \emph{Monetary Policy}: The process by which the central bank controls the money supply, typically through adjustments to interest rates.
    \item \emph{Inflation}: The rate at which the general level of prices for goods and services is rising, reflecting changes in purchasing power.
    \item \emph{Aggregate Demand}: The total demand for goods and services in an economy at a given overall price level and period.
    \item \emph{Food Prices}: The market prices of food products.
    \item \emph{Industrial Goods Prices}: The prices of manufactured goods produced in industrial settings.
    \item \emph{Services Prices}: The prices charged for common services.
    \item \emph{Labor Market}: The interaction between workers and employers regarding employment, wages, and job availability.
    \item \emph{GDP Growth}: The rate of increase in a country's Gross Domestic Product, indicating economic expansion.
    \item \emph{Exchange rate (BRL)}: The value of the Brazilian Real in relation to foreign currencies.
    \item \emph{Trade Deficit}: A situation where a country's imports exceed its exports.
    \item \emph{Cattle Cycle}: The cyclical fluctuations in beef supply and prices driven by changes in cattle production.
\end{itemize}


\paragraph{Examples}
\begin{itemize} 
    \item “Copom unanimously decided to increase the Selic rate by 0.50 p.p. to 11.25\% p.a.”
    \\ \textbf{Hawkish:} Increasing the interest rate is a contractionary move aimed at curbing inflation, indicating that policymakers are concerned about an overheating economy or rising inflation pressures. 
    
    \item “Copom assessed that the scenario—marked by resilient economic activity, labor market pressures, a positive output gap, increased inflation projections, and deanchored expectations—requires a more contractionary monetary policy.”
    \\ \textbf{Hawkish:} The sentences describes a condition (a positive output gap and rising inflation expectations) that justify tightening monetary policy to prevent the economy from overheating. 
    
    \item “All members agreed that it was appropriate to reduce the Selic rate by 0.50 percentage points.”
    \\ \textbf{Dovish:} A reduction in the interest rate is an expansionary action designed to stimulate economic activity. It reflects a concern that the economy might be underperforming, requiring expansionary monetary policy. 
    
    \item “The combination of a robust labor market, expansionary fiscal policy, and vigorous lending to households continues to support consumption and, consequently, aggregate demand.”
    \\ \textbf{Dovish:} The sentence implies that the economy is receiving supportive stimulus from other sectors, reducing the immediate need for tighter monetary policy. Even though it recognizes strong economic activity, it suggests that current measures are sufficient without extra contractionary steps. 
    
    \item “The labor market, which surprised positively throughout 2022, has shown resilience, with a net increase in job creation and relative stability in the unemployment rate.”
    \\ \textbf{Neutral:} The sentence reports economic conditions—namely, stability in the labor market—without suggesting any particular policy action. It serves as an observation of the current economic state rather than a signal for change. 

    \item “However, most of the committee members judged that this interpretation still seems premature and needs further corroboration by data.”
    \\ \textbf{Irrelevant:} This statement mainly expresses skepticism toward an interpretation and does not signal any explicit monetary policy direction or stance.
\end{itemize}

\newpage

\begin{longtable}{p{0.118\textwidth}p{0.183\textwidth}p{0.183\textwidth}p{0.183\textwidth}p{0.183\textwidth}}
\caption{\mptitle{Central Bank of Brazil}} \\
\toprule
\textbf{Category} & \textbf{Hawkish} & \textbf{Dovish} & \textbf{Neutral} & \textbf{Irrelevant} \\
\midrule
\endfirsthead

\toprule
\textbf{Category} & \textbf{Hawkish} & \textbf{Dovish} & \textbf{Neutral} & \textbf{Irrelevant} \\
\midrule
\endhead
\textbf{Selic Rate} & Increase in Selic rate to combat inflation. & Decrease in Selic rate to combat inflation. & Selic rate remains unchanged. & Sentence is not relevant to monetary policy. \\
\midrule
\textbf{Monetary Policy} & Central bank institutes contractionary or tightening monetary policy. & Central bank institutes expansionary or loosens monetary policy. & Monetary policy remains unchanged. & Sentence is not relevant to monetary policy. \\
\midrule
\textbf{Inflation} & Inflation exceeds upper limit of the tolerance band. Increasing inflation, inflationary pressures or expectations of inflation. & Inflation drops below lower limit of tolerance band. Decreasing inflation, inflationary pressures or expectations of inflation. & Inflation is within the central bank’s target range, with inflation and expectations stable. COPOM’s tolerance range for inflation is between 2\% and 5\%, with the target in the middle. & Sentence is not relevant to monetary policy. \\
\midrule
\textbf{Aggregate Demand (AD)} & Increasing aggregate demand. & AD is insufficient or decreasing. & Balanced AD and Aggregate Supply (AS), stable inflation, and steady growth. & Sentence is not relevant to monetary policy. \\
\midrule
\textbf{Food Prices} & Increasing or high food prices. & Decreasing or low food prices. & No change in food prices. & Sentence is not relevant to monetary policy. \\
\midrule
\textbf{Industrial Goods Prices} & Increasing or high price of industrialized goods. & Decreasing or low price of industrialized goods. & No change in price of industrialized goods. & Sentence is not relevant to monetary policy. \\
\midrule
\textbf{Services Prices} & Increasing or high service prices. & Decreasing or low service prices. & No change in price of services. & Sentence is not relevant to monetary policy. \\
\midrule
\textbf{Labor Market} & Low or decreasing unemployment. Demand for workers exceeds available supply. Excessive wage growth. Creation of more jobs. & High or decreasing unemployment. Weak demand for labor; underemployment; stagnant wages. & Full employment without overheating, stable wage growth, and balanced labor supply and demand. Average unemployment is around 10\%. & Sentence is not relevant to monetary policy. \\
\midrule
\textbf{GDP Growth} & GDP grows too quickly, accelerating beyond the country’s sustainable capacity. & Weak, negative (recession), or insufficient GDP growth to sustain full employment and price stability. & Stable GDP growth without excessive fluctuations. COPOM examines the circumstances and global context; on average, growth is around 1\% quarterly. & Sentence is not relevant to monetary policy. \\
\midrule
\textbf{Exchange rate (BRL)} & When the BRL depreciates. & When the BRL appreciates. & When a foreign currency appreciates or depreciates. & Sentence is not relevant to monetary policy. \\
\midrule
\textbf{Trade Deficit} & Decrease in exports or increase in imports, leading to a higher trade deficit. & Increase in exports or decrease in imports, leading to a lower trade deficit. & N/A. & Sentence is not relevant to monetary policy. \\
\midrule
\textbf{Cattle Cycle} & Supply of beef is low, leading to higher prices. & Supply of beef is high, leading to lower prices. & N/A. & Sentence is not relevant to monetary policy. \\
\bottomrule
\label{tb:bcb_mp_stance_guide}
\end{longtable}

