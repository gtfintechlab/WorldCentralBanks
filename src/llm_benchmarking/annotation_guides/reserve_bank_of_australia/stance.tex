\mptext{rba}{eleven} Economic Status, Government Spending, Key Words and Phrases, Dollar Value Change, Energy, Commodity, House Prices, RBA Expectations, Actions, Assets, Foreign Nations, Money Supply, Labor, Consumer Sentiment, Wages.


\textbf{Definitions:}
\begin{itemize}
    \item \textit{Economic Status}: A sentence discussing the current state of inflation, unemployment, or overall economic growth, and how these indicators influence monetary policy decisions.

    \item \textit{Government Spending}: A sentence pertaining to levels of government spending, such as enacting expansionary or contractionary fiscal policy.

    \item \textit{Key Words and Phrases}: A sentence that contains key word or phrase that would classify squarely into one of the three label classes, based upon its frequent usage and meaning among particular label classes.

    \item \textit{Dollar Value Change}: A sentence pertaining to changes such as appreciation or depreciation of value of the Australian Dollar on the Foreign Exchange Market.

    \item \textit{Energy/Commodity/House Prices}: A sentence pertaining to Energy via changes in prices of energy commodities or the energy sector as a whole, Commodities via precious metals or agricultural goods, or House Prices via single-family homes or the real estate sector as a whole.

    \item \textit{RBA Expectations/Actions/Assets}: A sentence that discusses changes in the cash rate, bond value, reserves, or any other financial asset value.

    \item \textit{Foreign Nations}: A sentence pertaining to trade relations between Australia and a foreign country. If not discussing Australia we label neutral.

    \item \textit{Money Supply}: A sentence that overtly discusses impact to the money supply or changes in demand.

    \item \textit{Labor}: A sentence that relates to changes in labor productivity.

    \item \textit{Consumer Sentiment}: A sentence reflecting the general confidence of households regarding future economic conditions.

    \item \textit{Wages}: A sentence highlighting wage growth or decline, which impacts inflation and consumer spending
\end{itemize}




\paragraph{Examples: }
\begin{itemize}
    \item ``The AONIA remained below target despite a recent policy change.''\\
    \textbf{Dovish}: Indicates that lower-than-target overnight lending rates suggest an easing stance.
    
    \item ``The RBA’s stance has been accommodative toward recent inflationary pressures.''\\
    \textbf{Dovish}: Suggests that the central bank is prioritizing economic support over tightening measures.
    
    \item ``Inflation has proved to be persistent and a challenge, staying above the inflation target even with recent efforts.''\\
    \textbf{Hawkish}: Implies that ongoing high inflation may necessitate a tightening of monetary policy.
    
    \item ``Import trade restrictions for China, one of Australia’s key trading partners could lead to supply shortages and higher prices, consequentially increasing inflationary pressures.''\\
    \textbf{Hawkish}: Highlights how external trade barriers can drive inflation upward, justifying stricter policy.
    
    \item ``After consideration, the cash rate target will remain unchanged as of October 2024.''\\
    \textbf{Neutral}: Reflects a decision to maintain the current policy without alteration.

    \item ``Members recognized that the calibration of this guidance was not precise or straightforward.''\\
    \textbf{Irrelevant}: Discusses the clarity and formulation of policy guidance rather than actual economic indicators or monetary actions.
\end{itemize}


\newpage

\begin{longtable}{p{0.118\textwidth}p{0.183\textwidth}p{0.183\textwidth}p{0.183\textwidth}p{0.183\textwidth}}
\caption{\mptitle{Reserve Bank of Australia}} \label{tb:rba_mp_stance_guide} \\

\toprule
\textbf{Category} & \textbf{Dovish} & \textbf{Hawkish} & \textbf{Neutral} & \textbf{Irrelevant} \\
\midrule
\endfirsthead

\toprule
\textbf{Category} & \textbf{Dovish} & \textbf{Hawkish} & \textbf{Neutral} & \textbf{Irrelevant} \\
\midrule
\endhead

\textbf{Economic Status} & 
When inflation decreases, when economic growth is predicted to be low, when cash rate (the goal overnight lending rate) target will be lowered, when inflation is below the inflation target, when the AONIA (the overnight lending rate) is below the cash rate target. & 
When inflation increases, when economic growth is predicted to be high, when cash rate target will be raised, when inflation is above the inflation target, when the AONIA (aka cash rate) is above the cash rate target. & 
When inflation remains unchanged, when the cash rate target remains unchanged, when inflation is at the inflation target, when the cash rate is at the cash rate target. & Sentence is not relevant to monetary policy. \\
\midrule

\textbf{Government Spending} & 
When government spending decreases. & 
When government spending increases. & N/A & Sentence is not relevant to monetary policy. \\
\midrule

\textbf{Key Words and Phrases} & 
When the stance is “accommodative,” indicating a focus on “maximum employment” and “price stability.” & 
Indicating a focus on “price stability” and “sustained growth.” & 
Use of phrases “mixed,” “moderate,” “reaffirmed.” & Sentence is not relevant to monetary policy. \\
\midrule

\textbf{Dollar Value Change} & 
When the Australian Dollar appreciates. & 
When the Australian Dollar depreciates. & N/A & Sentence is not relevant to monetary policy. \\
\midrule

\textbf{Energy, Commodity, House Prices} & 
When energy, commodities, or home prices decrease. & 
When energy, commodities, or home prices increase. & N/A & Sentence is not relevant to monetary policy. \\
\midrule

\textbf{RBA Expectations, Actions, Assets} & 
RBA expects subpar inflation, RBA expecting disinflation, narrowing spreads of treasury bonds, decreases in treasury security yields, and reduction of bank reserves. & 
RBA expects high inflation, widening spreads of treasury bonds, increase in treasury security yields, increase bank reserves. & N/A & Sentence is not relevant to monetary policy. \\
\midrule

\textbf{Foreign Nations} & 
When Australia’s trade deficit decreases or there are positive developments in trade relations, especially with top trading partners such as China, Japan, US, and the EU. & 
When Australia’s trade deficit increases or trade relations worsen, especially concerning top trading partners such as China, Japan, US, and the EU. & 
When relating to a foreign nation’s policy with no effect. & Sentence is not relevant to monetary policy. \\
\midrule

\textbf{Money Supply} & 
Money supply is low, M2 increases, increased demand for loans. & 
Money supply is high, increased demand for goods, low demand for loans. & N/A & Sentence is not relevant to monetary policy. \\
\midrule

\textbf{Labor} & 
When productivity increases, when unemployment increases. & 
When productivity decreases, when unemployment decreases. & N/A & Sentence is not relevant to monetary policy. \\
\midrule

\textbf{Consumer Sentiment} & 
Negative Consumer Sentiment. & 
Positive Consumer Sentiment. & N/A & Sentence is not relevant to monetary policy. \\
\midrule

\textbf{Wages} & 
Wages decrease. & 
Wages increase. & N/A & Sentence is not relevant to monetary policy. \\
\bottomrule
\end{longtable}