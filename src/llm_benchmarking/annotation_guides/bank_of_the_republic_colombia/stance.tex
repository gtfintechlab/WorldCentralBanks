\mptext{banrep}{eight} Inflation, Employment, Growth, Peso Value Change, Trade, BanRep Interest Rate, Energy, and Climate.

\begin{itemize}
    \item \emph{Inflation}: How much inflation is present in the Colombian Economy, measured by CPI, and whether or not this value increases.
    \item \emph{Employment}: The rate of unemployment in Colombia and its change over time.
    \item \emph{Growth}: The rate of GDP growth in Colombia's economy, and any changes to its value.
    \item \emph{Peso Value Change}: The appreciation, depreciation, or maintenance of the Colombian Peso.
    \item \emph{Trade}: The amount of imports and exports going in, and any changes in the amounts and distribution of these goods.
    \item \emph{BanRep Interest Rate}: The current interest rate set by the Central Bank of Colombia and any decisions surrounding it.
    \item \emph{Energy}: The availability and price of energy in Colombia
    \item \emph{Climate}: Notable climate conditions impacting Colombia, such as El Ni\~no.
\end{itemize}

\paragraph{Examples: }
\begin{itemize}
    \item ``They noted that the concern of a significant impact on food or power costs by the El Ni\~no phenomenon has been waning.''\\
    \textbf{Dovish}: Poor climate conditions like El Ni\~no put inflationary pressures on agricultural and energy sectors. With these effects improving, there is less inflation and more room to cut rates to stimulate the economy.
    
    \item ``They cautioned against sudden and unexpected interest rate cuts, which could fuel inflation expectations and currency depreciation, potentially jeopardizing the approximation to the inflation target and necessitating reversals with significant credibility costs.''\\
    \textbf{Hawkish}: The BanRep committee members state their hesitancy toward cutting rates, indicating a hawkish outlook.
    
    \item ``In the case of electricity rates, they emphasize that the risk of power outages has dissipated as hydro generation has been increasing and new solar plants are coming online.'' \\
    \textbf{Dovish}: Better power infrastructure is less likely to put inflationary pressures on energy prices, reducing the need to increase rates.

    \item ``This trend persists because essential items like rents, education, and personal services, among others, tend to adjust in line with observed inflation and wage increments.'' \\
    \textbf{Neutral}: Makes a neutral remark regarding observed trends surrounding inflation, but does not provide a reason to either cut or raise rates.

    \item ``A regular meeting of the board of directors of Banco de la Republica was held in the city of Bogota on may 23, 2008.''\\
    \textbf{Irrelevant}: This sentence is a statement about an administrative meeting and does not contain any monetary policy stance.
\end{itemize}

\newpage

\begin{longtable}{p{0.118\textwidth}p{0.183\textwidth}p{0.183\textwidth}p{0.183\textwidth}p{0.183\textwidth}}
\caption{\mptitle{Bank of the Republic}} \\
\toprule
\textbf{Category} & \textbf{Dovish} & \textbf{Hawkish} & \textbf{Neutral} & \textbf{Irrelevant} \\
\midrule
\endfirsthead

\toprule
\textbf{Category} & \textbf{Dovish} & \textbf{Hawkish} & \textbf{Neutral} & \textbf{Irrelevant} \\
\midrule
\endhead
\textbf{Inflation} & When inflation is declining or is below the 3\% target. & When inflation is rising or is above the 3\% target. & When inflation is close to the target and unchanged. & Sentence is not relevant to monetary policy. \\
\midrule
\textbf{Employment} & Low/decreasing employment, deterioration of job quality, or sectors lobbying for rate cuts to improve the labor market. & High/increasing employment or improvements in the labor market. & Stable unemployment rates. & Sentence is not relevant to monetary policy. \\
\midrule
\textbf{Economic Growth} & When economic growth is currently low or expected to be low in the future. & When economic growth is currently too high to be sustainable, or expected to be as such. & When economic growth is stable without causing economic overheating. & Sentence is not relevant to monetary policy. \\
\midrule
\textbf{Peso Value Change} & When the Peso appreciates. & When the Peso depreciates. & N/A & Sentence is not relevant to monetary policy. \\
\midrule
\textbf{Trade} & More exports or fewer imports, leading to a lower trade deficit. & Less exports or more imports, leading to a higher trade deficit. & N/A & Sentence is not relevant to monetary policy. \\
\midrule
\textbf{BanRep Interest Rate} & When BanRep decides to or is planning on lowering the interest rate. & When BanRep decides to or is planning on raising the interest rate. & When BanRep leaves interest rate unchanged. & Sentence is not relevant to monetary policy. \\
\midrule
\textbf{Energy} & Falling energy prices, higher energy availability. & Rising energy prices, lower energy availability. & N/A & Sentence is not relevant to monetary policy. \\
\midrule
\textbf{Climate} & When climate in the region improves, reducing agricultural risks leading to inflation (for examples, food supply and prices). & When poor climate and weather conditions increase food prices or reduce agricultural output, leading to inflation. & N/A & Sentence is not relevant to monetary policy. \\
\midrule
\textbf{Investment} & Lower levels of investment or delays in investment. & Higher levels of investment. & N/A & Sentence is not relevant to monetary policy. \\
\midrule
\textbf{Security Prices} & Higher prices of public debt. & Lower prices of public debt. & N/A & Sentence is not relevant to monetary policy. \\
\midrule
\textbf{Credit} & Decrease in credit or low levels of credit growth. & When credit growth is high. & N/A & Sentence is not relevant to monetary policy. \\
\bottomrule
\label{tb:banrep_mp_stance_guide}
\end{longtable}
