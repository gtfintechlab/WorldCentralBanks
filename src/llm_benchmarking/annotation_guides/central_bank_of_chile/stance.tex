\mptext{cboc}{nine} Inflation, Economic Growth, Copper Prices, Peso Exchange Rate, Foreign Reserves, External Demand (Trade Partners), Key Words/Phrases, Labor Market, Energy Prices. 

\begin{itemize}
    \item \emph{Inflation}: A sentence relating to the monetary policy decision of the Central Bank of Chile.
    
    \item \emph{Economic Growth}: A sentence pertaining to the state of the economy, relating to unemployment, investment, and inflation.
    
    \item \emph{Copper Prices}: A sentence relating to changes in copper prices, since Chile is the largest copper exporter and therefore, is an economic indicator for Chile.
    
    \item \emph{Peso Exchange Rate}: A sentence pertaining to changes in the peso, such as appreciation or depreciation of the value of the Peso on the Foreign Exchange Market.
    
    \item \emph{Foreign Reserves}: A sentence reflecting the central bank’s stance on liquidity and currency stabilization.
    
    \item \emph{External Demand}: Since Chile’s economy is export-dependent, a sentence pertaining to global demand becomes a critical factor.
    
    \item \emph{Key Words/Phrases}: A sentence that contains a key word or phrase that would classify it squarely into one of the three label classes, based upon its frequent usage and meaning among particular label classes.
    
    \item \emph{Labor Market}: A sentence that relates to changes in labor productivity.
    
    \item \emph{Energy Prices}: Lower energy prices reduce inflation and production costs, while higher energy prices increase them, prompting different monetary responses from the Central Bank.
\end{itemize}

\paragraph{Examples: }
\begin{itemize}
    \item ``Meanwhile, the longer-term rates had risen less than their external counterparts, in a context in which the local risk indicators remained contained.''\\
    \textbf{Neutral}: This statement only describes the market conditions without indicating a tightening or easing policy stance. 
    
    \item ``The median of the Financial Traders Survey (FTS) remained around 3.5\% for the third month in a row, while 90\% of the respondents expected inflation to be higher than 3\%.''\\
    \textbf{Hawkish}: The fact that 90\% of the respondents believed that the inflation was above 3\% suggests that there are some inflationary pressures.
    
    \item ``In this context, a major challenge would be to maintain the impulse of monetary and fiscal policy, as well as the ability to safeguard financial stability, until the economy was able to achieve a self-sustained growth rate, reducing the gaps and frictions that still remained.''\\
    \textbf{Dovish}: The statement suggests a goal to maintain stability which suggests a more accommodative approach.
    
    \item ``All five board members agreed that, from the analysis of the background information submitted in the preparation of the March MP report, it could be concluded that there was still no evidence of consolidation of said inflationary convergence.''\\
    \textbf{Hawkish}: In this sentence, it was evident from the board members' agreement that the inflation has not yet stabilized, signaling that inflation is too high or volatile. Thus, potentially requiring tightening monetary policy.
    
    \item ``There were also differences across multiple sources of information.''\\
    \textbf{Irrelevant}: This sentence points out that differences exist across information sources but is ultimately carry any monetary policy stance.
\end{itemize}


\clearpage


\begin{longtable}{p{0.118\textwidth}p{0.183\textwidth}p{0.183\textwidth}p{0.183\textwidth}p{0.183\textwidth}}
\caption{\mptitle{Central Bank of Chile}} \label{tb:cboc_mp_stance_guide} \\
\toprule
\textbf{Category} & \textbf{Hawkish} & \textbf{Dovish} & \textbf{Neutral} & \textbf{Irrelevant} \\
\midrule
\endfirsthead

\toprule
\textbf{Category} & \textbf{Hawkish} & \textbf{Dovish} & \textbf{Neutral} & \textbf{Irrelevant} \\
\midrule
\endhead

\textbf{Inflation} & Sentences indicating inflation significantly below the 2\% target suggest concerns over deflation or weak demand. & Sentences where inflation exceeds 4\% highlight the necessity for tighter monetary policies to curb inflationary pressures. & Sentences indicating inflation rates within the 2.5\% to 3.5\% ranges suggest stable price levels with no immediate need for policy adjustments. & Sentence is not relevant to monetary policy. \\
\midrule

\textbf{Economic Growth} & Descriptions of GDP growth below 2\% or unemployment over 8\% indicate the need for economic stimulus. & Descriptions of GDP growth above 5\% and unemployment below 4\% suggest an overheating economy requiring cooling measures. & Sentences indicating GDP growth between 3\% and 4\%, and unemployment rates from 4\% to 6\%, represent balanced economic conditions. & Sentence is not relevant to monetary policy. \\
\midrule

\textbf{Copper Prices} & Statements that copper prices are below approximately \$7,000/ton indicate reduced export revenues and potential economic strain. & References to copper prices exceeding \$10,000/ton suggest potential inflationary pressures, possibly necessitating tighter monetary policies. & Descriptions of copper prices within \$7,500 to \$9,500/ton indicate stable conditions conducive to current policy settings. & Sentence is not relevant to monetary policy. \\
\midrule

\textbf{Peso Exchange Rate} & Any mention of the peso depreciating more than 5\% suggests potential harm to export competitiveness, which might call for central bank intervention. & Mentions of the peso depreciating more than 5\%-10\% in a short period, pointing to possible inflation spikes and the need for a tighter monetary stance. & Statements indicating minor fluctuations (less than 2\% per month) suggest a stable exchange rate. & Sentence is not relevant to monetary policy. \\
\midrule

\textbf{Foreign Reserves} & Rapid increases in reserves (more than 5\% per month) can indicate an intention to inject liquidity into the economy. & Decreases in reserves (more than 5\%) can suggest a tightening of monetary liquidity. & Moderate changes (within 2\% month-over-month) in reserves indicate stability. & Sentence is not relevant to monetary policy. \\
\midrule

\textbf{External Demand (Trade Partners)} & Sentences indicating a weakening in external demand, especially from major trade partners like China, suggest economic downturns. & Increased global demand for Chilean exports, as mentioned in statements, indicates robust economic health. & Stable external demand levels imply a steady economic environment. & Sentence is not relevant to monetary policy. \\
\midrule

\textbf{Key Words and Phrases} & Sentences focusing on “stimulating growth,” “accommodative monetary policy,” or “reducing unemployment” are categorized as dovish. & Phrases like “inflation control,” “price stability,” or “reducing liquidity” are considered hawkish. & Neutral terms include “balanced,” “steady,” or “in line with expectations.” & Sentence is not relevant to monetary policy. \\
\midrule

\textbf{Labor Market} & References to high unemployment rates (over 8\%) and stagnant wage growth (less than 2\%) indicate a need for supportive monetary policies. & Low unemployment (less than 4\%) and rising wages (over 5\%) suggest an overheating labor market that may need cooling measures. & Descriptions of unemployment rates between 4\%-6\% and wage growth between 3\%-4\% indicate a healthy labor market. & Sentence is not relevant to monetary policy. \\
\midrule

\textbf{Energy Prices} & Statements of energy prices declining by more than 10\% often suggest decreasing inflationary pressures and reduced production costs. & Descriptions of energy prices increasing by more than 10\% indicate rising inflation pressures, potentially requiring tighter monetary measures. & Sentences noting energy price changes within a 1\%-2\% range month-over-month suggest price stability. & Sentence is not relevant to monetary policy. \\
\bottomrule
\end{longtable}
