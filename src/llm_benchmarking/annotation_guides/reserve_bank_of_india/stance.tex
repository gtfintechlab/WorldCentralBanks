\mptext{rbi}{fourteen} Inflation, Repo Rate, Reverse Repo Rate, Cash Reserve Ratio, Statutory Liquidity Ratio, GDP Growth Forecast, Monetary Policy Measures, Export Performance, Manufacturing Activity, Consumer Demand, Employment Levels, Commodity Prices, Credit Growth, Exchange Rate:

\begin{itemize}
    \item \emph{Inflation}: A sentence pertaining to the general increase in prices for goods and services; the decrease in the purchasing power of a currency.
    \item \emph{Repo Rate}: A sentence pertaining to the interest rate at which the central bank lends money to commercial banks.
    \item \emph{Reverse Repo Rate}: A sentence pertaining to the rate at which the central bank borrows funds from commercial banks to manage liquidity.
    \item \emph{Cash Reserve Ratio (CRR)}: A sentence pertaining to the percentage of a bank's deposits that must be held in reserve and not lent out.
    \item \emph{Statutory Liquidity Ratio (SLR)}: A sentence pertaining to the minimum percentage of a bank's liabilities that must be held in the form of liquid assets.
    \item \emph{GDP Growth Forecast}: A sentence pertaining to the projected rate of increase in a country’s Gross Domestic Product, indicating economic expansion.
    \item \emph{Monetary Policy Measures}: A sentence pertaining to actions taken by the central bank to influence the money supply and interest rates.
    \item \emph{Export Performance}: A sentence pertaining to the export activity of a country in foreign markets.
    \item \emph{Manufacturing Activity}: A sentence pertaining to the level of production and industrial output within the manufacturing sector.
    \item \emph{Consumer Demand}: A sentence pertaining to the overall spending by households on goods and services.
    \item \emph{Employment Levels}: A sentence pertaining to job availability, unemployment, and overall labor market conditions.
    \item \emph{Commodity Prices}: A sentence pertaining to the market prices of primary raw materials and inputs used in production.
    \item \emph{Credit Growth}: A sentence pertaining to the rate at which bank lending increases, reflecting prevailing financial conditions.
    \item \emph{Exchange Rate Movements}: A sentence pertaining to fluctuations in the value of a country’s currency relative to other currencies.
\end{itemize}


\paragraph{Examples}
\begin{itemize}
    \item “With persistently high food inflation, it would be in order to continue with the disinflationary policy stance that we have adopted.” \\
    \textbf{Hawkish:} Increasing the focus on curbing high inflationary pressures indicates a tightening stance aimed at controlling rising prices.
    
    \item “I am, therefore, of the view that a reduction in the policy repo rate by conventional 25 bps will be inadequate.” \\
    \textbf{Hawkish:} This sentence suggests that a minor rate cut is insufficient to address economic challenges, hinting at the need for stronger tightening measures.
    
    \item “There has also been an inching down in the median 3-month and 1-year ahead inflation expectations which is also comforting.” \\
    \textbf{Dovish:} The observation of falling inflation expectations supports a dovish stance, where easing measures are seen as appropriate given the lower inflation outlook.
    
    \item “Export dependent industries such as textiles are not doing well.” \\
    \textbf{Dovish:} Highlighting weak export performance implies that easing measures might be necessary to stimulate growth in export sectors.
    
    \item “Since the Indian middle income consumer is price sensitive, profits have risen.” \\
    \textbf{Neutral:} This sentence provides an observation on market conditions without signaling a clear need for either tightening or easing monetary policy.

    \item “It need not be a concern for the MPC.” \\
    \textbf{Irrelevant:} This statement mentioned the importance of a statement and does not discuss monetary policy or take a monetary policy stance.
\end{itemize}

\begin{longtable}{p{0.118\textwidth}p{0.183\textwidth}p{0.183\textwidth}p{0.183\textwidth}p{0.183\textwidth}}
\caption{\mptitle{Reserve Bank of India}} \label{tb:rbi_mp_stance_guide}
\\
\toprule
\textbf{Category Term} & \textbf{Hawkish} & \textbf{Dovish} & \textbf{Neutral} & \textbf{Irrelevant} \\
\midrule
\endfirsthead

\toprule
\textbf{Category} & \textbf{Hawkish} & \textbf{Dovish} & \textbf{Neutral} & \textbf{Irrelevant} \\
\midrule
\endhead

\textbf{Inflation} & Inflation above target or increasing; rising inflationary pressures prompting tighter policy. & Inflation below target or decreasing; low inflation encouraging easing to support growth. & Inflation at target or stable with no policy change. & Sentence is not relevant to monetary policy. \\
\midrule
\textbf{Repo Rate} & Increasing the repo rate to tighten monetary policy and curb inflation. & Decreasing the repo rate to ease monetary conditions and stimulate economic activity. & Maintaining the repo rate, reflecting balanced conditions. & Sentence is not relevant to monetary policy. \\
\midrule
\textbf{Reverse Repo Rate} & Increasing the reverse repo rate to reduce liquidity by encouraging banks to park funds. & Decreasing the reverse repo rate to boost liquidity by incentivizing lending. & Maintaining the reverse repo rate, indicating steady liquidity management. & Sentence is not relevant to monetary policy. \\
\midrule
\textbf{Cash Reserve Ratio (CRR)} & Increasing CRR to reduce liquidity by requiring banks to hold more reserves. & Decreasing CRR to increase liquidity by allowing more lending. & Maintaining CRR, showing unchanged reserve requirements. & Sentence is not relevant to monetary policy. \\
\midrule
\textbf{Statutory Liquidity Ratio (SLR)} & Increasing SLR to force banks to hold more liquid assets, reducing lending capacity. & Decreasing SLR to allow banks to hold fewer liquid assets, increasing lending capacity. & Maintaining SLR, indicating unchanged liquidity constraints. & Sentence is not relevant to monetary policy. \\
\midrule
\textbf{GDP Growth Forecast} & Projecting lower GDP growth that may signal the need for tighter policy if inflation risks persist. & Projecting higher GDP growth that may lead to easing measures to support expansion. & Projecting stable GDP growth, supporting maintenance of current policy stance. & Sentence is not relevant to monetary policy. \\
\midrule
\textbf{Monetary Policy Measures} & Implementing tightening measures (e.g., raising rates) to combat inflation. & Implementing accommodative measures (e.g., lowering rates) to foster growth. & No significant policy changes, suggesting a continuation of current stance. & Sentence is not relevant to monetary policy. \\
\midrule
\textbf{Exports} & Strong export growth that could overheat the economy and prompt tighter policy. & Weak export performance that may require easing measures to stimulate growth. & Stable export levels with consistent foreign demand, supporting current policy. & Sentence is not relevant to monetary policy. \\
\midrule
\textbf{Manufact-uring Activity} & Rapid expansion that may trigger inflationary pressures, calling for tightening. & Decline in activity suggesting easing measures to revive industrial growth. & Steady, balanced manufacturing output with no policy change indicated. & Sentence is not relevant to monetary policy. \\
\midrule
\textbf{Consumer Demand} & Surging demand that might cause demand-pull inflation and necessitate tightening. & Weak consumer demand indicating low spending, warranting easing measures. & Sustainable, balanced consumer demand with no inflationary impact. & Sentence is not relevant to monetary policy. \\
\midrule
\textbf{Employment} & Very low unemployment that could lead to wage inflation and overheating, prompting tightening. & High unemployment suggesting the need for expansionary policies to stimulate job creation. & Stable employment with balanced growth, indicating no policy shift. & Sentence is not relevant to monetary policy. \\
\midrule
\textbf{Commodity Prices} & Rising prices that can drive cost-push inflation, necessitating tightening measures. & Falling prices reducing inflationary pressures, supporting a dovish stance. & Stable prices contributing to predictable costs, maintaining current policy stance. & Sentence is not relevant to monetary policy. \\
\midrule
\textbf{Credit Growth} & Rapid credit growth that might lead to financial instability and inflation, requiring tightening. & Slow credit growth indicating tight financial conditions, suggesting easing to encourage lending. & Moderate credit growth in line with expansion, indicating no change in policy stance. & Sentence is not relevant to monetary policy. \\
\midrule
\textbf{Exchange Rate Movements} & A depreciating currency increasing import costs and inflation, possibly calling for tightening. & An appreciating currency reducing import costs and inflation, allowing for easing. & Stable exchange rates that keep external pressures in check, maintaining neutrality. & Sentence is not relevant to monetary policy. \\
\bottomrule
\end{longtable}

