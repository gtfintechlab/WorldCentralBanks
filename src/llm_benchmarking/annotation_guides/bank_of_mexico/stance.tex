\mptext{bom}{7} Peso Value Change, Trade Activity, Labor, Key Words, Economic Growth,  Market Sentiment, Energy/House Prices.

\begin{itemize}
    \item \textit{Peso Value Change:}  
    Sentences corresponding to peso value changes such as inflation and/or foreign market appreciation/depreciation. A weaker peso may be seen as dovish, especially when Banco de México is attempting to support exports by making Mexican goods cheaper in foreign markets. A stronger peso helps control inflation by making imports cheaper, but it can hurt export competitiveness.

    \item \textit{Trade Activity:}  
    Sentences pertaining to the import/export ratio—an increase in one or decrease of another. Banco de México favors policies that stimulate exports, especially given Mexico’s strong manufacturing sector and trade agreements (such as USMCA). A focus on export growth could support job creation and help maintain balance in current account deficits.

    \item \textit{Labor:}  
    Sentences mentioning statistics about the increase or decrease in productivity. If there are concerns about declining productivity—especially in key industries in Mexico like oil, manufacturing, or agriculture—Banco de México might lean hawkish, signaling the need to cool off the economy to address inefficiencies.

    \item \textit{Key Words:}  
    Sentences that have keywords that make a clear, strong indication towards classifications.

    % \item \textbf{Interest Rates:}  
    % Sentences mentioning a change in interest rates or stating that there will be no change. A hawkish stance would likely aim at curbing inflation, which has been a persistent challenge for Mexico. In this case, higher interest rates may be designed to stabilize the peso or protect purchasing power.

    \item \textit{Economic Growth:}  
    Sentences conveying the performance in economic growth the bank wishes to see in the near future. Given Mexico’s economic structure, fostering growth in sectors like manufacturing, tourism, and remittances can be seen as a dovish move aimed at reducing poverty and unemployment.

    % \item \textbf{Fiscal Spending:}  
    % Sentences mentioning how the government is changing its spending policies. Banco de México may support higher government spending on social programs, particularly during economic downturns or times of political instability. This would reflect a dovish stance aimed at alleviating inequality or stimulating domestic demand.

    \item \textit{Market Sentiment:}  
    Sentences mentioning what the outlook is for the market and how investors are acting.

    \item \textit{Energy/House Prices:}  
    Sentences that mention anything about housing price levels and energy price levels. Given Mexico’s reliance on oil and gas revenues, a dovish outlook might stem from efforts to lower energy prices domestically, encouraging consumption and reducing inflationary pressure on households.

    % \item \textbf{Outlook:}  
    % This focuses on whether a sentiment is forward-looking (future) or not (past or present). This category is relatively self-explanatory and can usually be determined by the verb tense of the sentence.

    % \item \textbf{Certainty:}  
    % This focuses on whether the sentiment is sure something will happen or has happened. Certainty can usually be determined by key words in the sentence.
\end{itemize}


\paragraph{Examples:}

\begin{itemize}
    \item ``Another member noted that non-core inflation has been below forecasts.''\\ 
    \textbf{Dovish}: This sentence states that non-core inflation is lower than expected, possibly opening up opportunities for expansionary monetary policy.
    
    \item ``However, another member specified that breakeven inflation still remains at levels above 4\% for all terms.''\\ 
    \textbf{Hawkish}: This sentence highlights that breakeven inflation exceeds 4\% across all terms, hinting at potential future tightening to counter inflation concerns.
    
    \item ``In this context, some members noted that the confidence of economic agents decreased.''\\ 
    \textbf{Dovish}: This sentence suggests that a decline in economic confidence might encourage a more supportive monetary policy approach to improve confidence.

    \item ``Another member stated that the dynamics within the components of inflation has led to an upward adjustment of forecasts.''\\ 
    \textbf{Hawkish}: This sentence asserts that specific dynamics have already resulted in a revision upward of forecasts, requiring contractionary monetary policy to address inflation concerns.
    
    \item ``Some members considered that the economy's cyclical position have not changed significantly since the previous monetary policy decision.''\\ 
    \textbf{Neutral}: This sentence indicates that the economy’s cyclical stance remains stable, implying no immediate change in policy direction.
    
    \item ``One member added that the bank of Canada is also preparing to do so.''\\ 
    \textbf{Irrelevant}: This sentence refers to potential preparatory actions by the Bank of Canada without directly addressing a monetary policy decision or stance.
\end{itemize}


\newpage


\begin{longtable}{p{0.118\textwidth}p{0.183\textwidth}p{0.183\textwidth}p{0.183\textwidth}p{0.183\textwidth}}
\caption{\mptitle{Bank of Mexico}} \label{tb:bom_mp_stance_guide} \\
\toprule
\textbf{Category} & \textbf{Dovish} & \textbf{Hawkish} & \textbf{Neutral} & \textbf{Irrelevant} \\
\midrule
\endfirsthead

\toprule
\textbf{Category} & \textbf{Dovish} & \textbf{Hawkish} & \textbf{Neutral} & \textbf{Irrelevant} \\
\midrule
\endhead
\textbf{Peso Value Change} & Peso appreciates, leading to lower inflation. & Peso depreciates, leading to higher inflation. & N/A & Sentence is not relevant to monetary policy. \\
\midrule
\textbf{Trade Activity} & Favors decreased imports and increased exports, supporting domestic growth. & Favors increased imports and decreased exports, raising dependency on foreign markets. & N/A & Sentence is not relevant to monetary policy. \\
\midrule
\textbf{Labor} & Favors increased labor productivity, improving economic output. & Favors lower productivity, signaling economic slowdown. & Labor market remains stable, no notable changes. & Sentence is not relevant to monetary policy. \\
\midrule
\textbf{Key Words} & Words like “Accommodative”, “Expansionary”, “Easing”, indicating support for stimulus. & Words like “Tighter”, “Restrictive”, “Balance”, indicating a shift toward tightening. & Words like “Mixed”, “Moderate”, signaling a neutral stance. & Sentence is not relevant to monetary policy. \\
\midrule
\textbf{Economic Growth} & Supports expansion, higher consumer spending, and investment. & Advocates policies leading to economic slowdown or controlled growth. & Economy remains balanced, no significant shifts in output. & Sentence is not relevant to monetary policy. \\
\midrule
\textbf{Market Sentiment} & Bullish outlook, rising investor confidence in the economy. & Bearish outlook, increased caution among investors. & Neutral market stance, adopting a wait-and-see approach. & Sentence is not relevant to monetary policy. \\
\midrule
\textbf{Energy/House Prices} & Rising property values and lower energy costs, benefiting consumers. & Housing market slowdown, rising energy costs, increasing financial burden. & N/A & Sentence is not relevant to monetary policy. \\
\bottomrule
\end{longtable}