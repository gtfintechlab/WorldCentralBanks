\mptext{bok}{14}: Economic Status, Keywords/Phrases, KRW Value Change, Housing Debt/Loan, House Prices, Foreign Nations, BOK Expectations/Actions/Assets, Money Supply, Labor, Interest rate gap between South Korea and United States, US Dollar Value, FX Market, COVID-19, and Energy Prices.

\begin{itemize}
    \item \textit{Economic Status}: A sentence pertaining to the state of the economy, relating to unemployment and inflation.
    \item \textit{Keywords/Phrases}: A sentence that contains key word or phrase that would classify it squarely into one of the three label classes, based upon its frequent usage and meaning among particular label classes.
    \item \textit{KRW Value Change}: A sentence pertaining to changes such as appreciation or depreciation of value of the South Korea Won on the Foreign Exchange Market.
    \item \textit{Housing Debt/Loan}: A sentence that relates to changes in debt in households.
    \item \textit{House Prices}: A sentence pertaining to changes in prices of real estate.
    \item \textit{Foreign Nations}: A sentence pertaining to trade relations between South Korea and a foreign country. If not discussing South Korea we label neutral.
    \item \textit{BOK Expectations/Actions/Assets}: A sentence that discusses changes in BOK yields, bond value, reserves, or any other financial asset value.
    \item \textit{Money Supply}: A sentence that overtly discusses impact to the money supply or changes in demand.
    \item \textit{Labor}: A sentence that relates to changes in labor productivity.
    \item \textit{Interest rate gap between South Korea and United States}: A sentence pertaining to changes such as increase or decrease of the interest rate gap between the United States and South Korea.
    \item \textit{US Dollar Value}: A sentence about fluctuation of the dollar’s value which can influence monetary policy in a dovish or hawkish direction.
    \item \textit{FX Market}: A sentence about stability’s changes in the foreign exchange market.
    \item \textit{COVID-19}: A sentence indicating whether the government is tightening or easing COVID policy.
    \item \textit{Energy Prices}: A sentence pertaining to changes in prices of energy commodities or the energy sector as a whole.
\end{itemize}



\paragraph{Examples: }
\begin{itemize}
    \item ``Concerning prices, the member presented the view that a depreciation of the won could further fuel inflationary pressures.''\\
    \textbf{Hawkish}: This statement warns that a depreciation of the won may lead to increased inflation, suggesting the need for a tighter monetary policy.
    
    \item ``Some of the members emphasized that it would be necessary to prepare against growing instability of the external sector due to the worsening global financial conditions including the rising won-dollar exchange rate, the narrowing current account surplus, and a possible foreign capital outflow owing to the widening gap between domestic and overseas interest rates.''\\
    \textbf{Hawkish}: This sentence reflects a hawkish stance by highlighting external financial instability and recommending proactive measures to counter potential inflationary pressures.
    
    \item ``Finally, in terms of financial stability, the member assessed that growth in household lending and housing prices had slowed since the second half of last year, but continued caution about the risk of a buildup of financial imbalances was required.''\\
    \textbf{Dovish}: This statement is dovish as it underscores caution and a preference for maintaining accommodative measures despite slowing growth in household lending and housing prices.
    
    \item ``In light of the narrowing interest rate gap between South Korea and the United States, some members suggested that the Bank of Korea may consider keeping interest rates low to support domestic economic recovery and encourage borrowing.''\\
    \textbf{Dovish}: This sentence advocates for lower interest rates to stimulate economic recovery, which is an expansionary monetary policy stance.

    \item ``The member, however, pointed out that the price path had not been showing a commensurate change.''\\
    \textbf{Neutral}: This sentence notes that the price path has not changed significantly without implying any particular monetary policy stance.
    
    \item ``The member thus emphasized the need to pay attention to this issue.''\\
    \textbf{Irrelevant}: This sentence is irrelevant because it does not provide enough context related to economic policy or market conditions relevant to the Bank of Korea.
\end{itemize}


\newpage


\begin{longtable}{p{0.118\textwidth}p{0.183\textwidth}p{0.183\textwidth}p{0.183\textwidth}p{0.183\textwidth}}
\caption{Bank of Korea Annotation Guide} \\
\toprule
\textbf{Category} & \textbf{Hawkish} & \textbf{Dovish} & \textbf{Neutral} & \textbf{Irrelevant} \\
\midrule
\endfirsthead

\toprule
\textbf{Category} & \textbf{Hawkish} & \textbf{Dovish} & \textbf{Neutral} & \textbf{Irrelevant} \\
\midrule
\endhead
\textbf{Economic Status} & When inflation decreases, when unemployment increases, when economic growth is projected as low. & When inflation increases, when unemployment decreases when economic growth is projected high when economic output is higher than potential supply/actual output when economic slack falls. & When unemployment rate or growth is unchanged, maintained, or sustained. & Sentence is not relevant to monetary policy. \\ 
\midrule
\textbf{KRW Value Change} & When KRW value appreciates. & When KRW value depreciates. & N/A & Sentence is not relevant to monetary policy. \\
\midrule
\textbf{US Dollar Value} & When USD is moderate or weakening. & When USD becomes stronger, as this indicates potential depreciation of KRW. & N/A & Sentence is not relevant to monetary policy. \\
\midrule
\textbf{Interest rate gap between South Korea and United States} & When US interest rate is aligned with South Korea’s, US not interfering with South Korea’s interest rates. & When US interest rate is higher than South Korea’s, potentially pressuring South Korea to raise interest rates. & N/A & Sentence is not relevant to monetary policy. \\
\midrule
\textbf{FX market} & Stable FX market. & Unstable FX market. & N/A & Sentence is not relevant to monetary policy. \\
\midrule
\textbf{Housing Debt/Loan} & When public or private debt level decreases, signaling more flexibility for borrowing and fiscal stimulus. & When public or private debt level decreases, signaling more flexibility for borrowing and fiscal stimulus. & N/A & Sentence is not relevant to monetary policy. \\
\midrule
\textbf{Housing Prices} & When house prices decrease or expected to decrease. & When house prices increase or expected to increase. & N/A & Sentence is not relevant to monetary policy. \\
\midrule
\textbf{Foreign \newline Nations} & When the Korean trade deficit decreases. & When the Korean trade deficit increases. & When relating to a foreign nation’s economic or trade policy. & Sentence is not relevant to monetary policy. \\
\midrule
\textbf{BOK Expectations/Actions/\newline Assets} & BOK expect supbar inflation, BOK expecting disinflation, narrowing spreads of treasury bonds, decreases in treasury security yields, and reduction of bank reserves. & When BOK signals the need for tightening, raising interest rates, or increasing reserves to curb inflation. & N/A & Sentence is not relevant to monetary policy. \\
\midrule
\textbf{Keywords/\newline Phrases} & When the stance is "accommodative", indicating a focus on “maximum employment” and “price stability.” & Indicating a focus on “price stability” and “sustained growth.” & Use of phrases “mixed”, “moderate”, “reaffirmed.” & Sentence is not relevant to monetary policy. \\
\midrule
\textbf{Money \newline Supply} & Money supply is low, M2 increases, increased demand for loans. & Money supply is high, increased demand for goods, low demand for loans. & N/A & Sentence is not relevant to monetary policy. \\
\midrule
\textbf{Labor} & When productivity increases, solid labor market. & When productivity decreases, unstable labor market. & N/A & Sentence is not relevant to monetary policy. \\
\midrule
\textbf{COVID-19} & Easing COVID-19 regulation: boost tourism industry and increased productivity. & Increased COVID-19 regulation: detrimental for tourism industry and decreased productivity. & N/A & Sentence is not relevant to monetary policy. \\
\midrule
\textbf{Energy} & When oil/energy prices decrease. & When oil/energy prices increase. & N/A & Sentence is not relevant to monetary policy. \\
\bottomrule
\label{tb:bok_mp_stance_guide}
\end{longtable}