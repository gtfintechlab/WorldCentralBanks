\mptext{boc}{eight} Inflation, Employment, Economic Growth, Interest Rates, Monetary Stimulus, Exchange Rates, Commodity Prices, and Fiscal Policy Interaction.


\begin{itemize} 
    \item \emph{Inflation}: A sentence pertaining to the evolution of price levels, encompassing trends of changing inflation rates or pressures on inflation. 
    \item \emph{Employment}: A sentence pertaining to labor market conditions, focusing on unemployment rates, job creation trends, and workforce strength. 
    \item \emph{Economic Growth}: A sentence pertaining to the pace and direction of economic expansion or contraction, including implications for stimulus during slowdowns or overheating during rapid growth. 
    \item \emph{Interest Rates}: A sentence pertaining to interest rate levels, discussing policy decisions aimed at modifying interest rates to address the needs of the economy. 
    \item \emph{Monetary Stimulus}: A sentence pertaining to actions that expand liquidity, such as increased asset purchases. 
    \item \emph{Exchange Rates}: A sentence pertaining to the movement of a nation's currency value or their economic implications. 
    \item \emph{Commodity Prices}: A sentence pertaining to shifts in the prices of Canada's commodities. 
    \item \emph{Fiscal Policy Interaction}: A sentence pertaining to the interactions between government fiscal measures (such as spending and taxation) and monetary policy decisions.
\end{itemize}

\paragraph{Examples: } 
\begin{itemize} 
    \item ``Inflation is expected to exceed 3\%, requiring tightening.''\\ 
    \textbf{Hawkish}: This sentence implies that, due to rising inflation above the target, the Bank of Canada may adopt tightening policies to control inflationary pressures.
    
    \item ``Unemployment has fallen significantly, prompting concerns about wage inflation.''\\
    \textbf{Hawkish}: This example indicates that a tight labor market may lead to wage pressures, suggesting that monetary policy could tighten to prevent an overheating economy.
    
    \item ``The central bank cut rates to support the slowing economy.''\\
    \textbf{Dovish}: This sentence states that the Bank of Canada took an accommodative stance to stimulate economic activity during economic slowdowns.
    
    \item ``Weak economic activity signals the need for further monetary stimulus.''\\
    \textbf{Dovish}: This example reflects the view that subdued growth conditions warrant more easing to boost economic performance.
    
    \item ``Interest rates remained steady this month.''\\
    \textbf{Neutral}: This sentence portrays a situation where the policy remains unchanged, reflecting stability without leaning towards tightening or easing.

    \item ``We will take decisions one meeting at a time.'' \\
    \textbf{Irrelevant}: This sentence discusses the way the Bank of Canada plans to make monetary policy decisions, however, it does not explicitly discuss any monetary policy action.
\end{itemize}


\newpage


\begin{longtable}{p{0.118\textwidth}p{0.183\textwidth}p{0.183\textwidth}p{0.183\textwidth}p{0.183\textwidth}}
\caption{Bank of Canada Annotation Guide} \label{tb:boc_mp_stance_guide} \\
\toprule
\textbf{Category} & \textbf{Dovish} & \textbf{Hawkish} & \textbf{Neutral} & \textbf{Irrelevant} \\
\midrule
\endfirsthead

\toprule
\textbf{Category} & \textbf{Dovish} & \textbf{Hawkish} & \textbf{Neutral} & \textbf{Irrelevant} \\
\midrule
\endhead

\textbf{Inflation} & 
Below target or deflation concerns leading to easing. & 
Above target or rising inflation, prompting tightening. & 
Descriptive with no policy implication. & 
Sentence is not relevant to monetary policy. \\
\midrule

\textbf{Employment} & 
High unemployment, weak labor market prompting easing. & 
Low unemployment, risk of overheating, tightening bias. & 
Labor description without policy implication. & 
Sentence is not relevant to monetary policy. \\
\midrule

\textbf{Economic Growth} & 
Slow growth or recession risk, stimulus needed. & 
Strong growth, risk of overheating. & 
Growth data with no implied action. & 
Sentence is not relevant to monetary policy. \\
\midrule

\textbf{Interest Rates} & 
Lowering or keeping rates low to support economy. & 
Raising rates to counter inflation. & 
Rate level mentioned without bias. & 
Sentence is not relevant to monetary policy. \\
\midrule

\textbf{Monetary Stimulus} & 
Increasing liquidity or asset purchases. & 
Reducing purchases or liquidity, signaling tightening. & 
Describes programs neutrally. & 
Sentence is not relevant to monetary policy. \\
\midrule

\textbf{Exchange Rates} & 
Depreciation prompts easing to support growth. & 
Appreciation prompts concern about overheating. & 
Neutral exchange rate statement. & 
Sentence is not relevant to monetary policy. \\
\midrule

\textbf{Commodity Prices} & 
Falling prices suggest weak demand, needing easing. & 
Rising prices suggest inflation, tightening needed. & 
Price movement with no policy stance. & 
Sentence is not relevant to monetary policy. \\
\midrule

\textbf{Fiscal Policy Interaction} & 
Monetary policy needed to support lagging fiscal stimulus. & 
Strong fiscal stimulus might require tightening. & 
Neutral fiscal commentary. & 
Sentence is not relevant to monetary policy. \\
\bottomrule
\end{longtable}
