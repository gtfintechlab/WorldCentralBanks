\mptext{cbr}{thirteen} Economic Status, Ruble Value Change, Oil/Energy Prices, House Prices, Agriculture/Food Prices, Foreign Nations \& Sanctions, Military/Defense, CBR Expectations/Actions/Assets, Inflation Rates (CPI), Money Supply, Key Rate Fluctuations, Key Words/Phrases, and Labor.
\begin{itemize}
    \item \emph{Economic Status}: A sentence pertaining to the state of the economy, relating to inflation levels or economic growth/decay. 
    \item \emph{Ruble Value Change}: A sentence pertaining to the appreciation, depreciation, or maintenance of the Ruble.
    \item \emph{Oil/Energy Prices}: A sentence that discusses fluctuations in Russian crude oil export prices, petroleum development industry performance, natural gas production and distribution, energy commodities, or the oil/energy sector as a whole within the Russian Federation. 
    \item \emph{House Prices}: A sentence pertaining to changes in prices of real estate, domestic/global housing demand, mortgage rates and rental market prices.
    \item \emph{Agriculture/Food Prices}: A sentence pertaining to the performance of the agricultural sector (crop yield, food exports) and/or fluctuations in food prices.
    \item \emph{Foreign Nations \& Sanctions}: A sentence pertaining to international trade relations and/or changes regarding Western sanctions on the Russian Federation and Russian response. 
    \item \emph{Military/Defense}: A sentence pertaining to defense spending and/or its effect on GDP, or about labor shortages exacerbated by wartime recruitment efforts. 
    \item \emph{CBR Expectations/Actions/Assets}: A sentence that discusses changes in the Bank of Russia's key rate, bond yields, foreign exchange reserves, or any other monetary policy instrument or financial asset value.
    \item \emph{Inflation Rates (CPI)}: A sentence regarding the inflation rate as measured by the Consumer Price Index (CPI) and any inflationary/dis-inflationary pressures. 
    \item \emph{Money Supply}: A sentence that overtly discusses impact to the money supply or changes in demand for the Russian Federation.
    \item \emph{Key Rate Fluctuations}: A sentence that describes decisions to manipulate the Key Rate and its corresponding fluctuations.
    \item \emph{Key Words/Phrases}: A sentence containing certain key words that would classify as hawkish, dovish, or neutral, based upon its frequency and sentiment.
    \item \emph{Labor}: A sentence that relates to changes in labor productivity of the Russian Federation.
\end{itemize}

\paragraph{Examples:}
\begin{itemize}
    \item ``In February—March 2019, inflation is holding somewhat lower than the Bank of Russia’s expectations.''\\
    \textbf{Dovish}: Low inflation as compared to expectations indicates a raise in price level is needed to meet the target, signaling that quantitative easing should be implemented to promote price growth. 
    
    \item ``Moreover, inflation expectations of households increased in July.''\\
    \textbf{Hawkish}: Inflationary expectations increasing signifies a potential need to implement tighter policies to bring inflation back down.

    \item ``The balance of risks remains skewed towards pro-inflationary risks, especially over a short-term horizon, driven by the VAT increase and price movements in individual food products.''\\
    \textbf{Hawkish}: Risks being aligned with elevated pro-inflationary risks over the short term indicates a need for immediate tightening policy to decrease pro-inflationary pressures.

    \item ``External inflation is stable and does not exert a noticeable influence on domestic prices.''\\
    \textbf{Neutral}: Inflation is stable and therefore no tightening to decrease inflation or easing to increase inflation is needed. 
    
    \item ``While assessing evolving inflation dynamics and economic developments against the forecast, the Bank of Russia admits the possibility of cutting the key rate gradually in coming Q2-Q3.''\\
    \textbf{Dovish}: Cutting the key rate to adjust inflation in line with the target projections is a form of monetary easing.
    
    \item ``On 27 October 2023, the Bank of Russia Board of Directors decided to increase the key rate by 200 basis points to 15.00\% per annum.''\\
    \textbf{Hawkish}: Increasing the key rate is a form of quantitative tightening and indicates a stricter monetary policy stance.
    
    \item ``On 3 February 2017, the Bank of Russia Board of Directors decided to keep the key rate at 10.00\% p.a.''\\
    \textbf{Neutral}: Maintaining the key rate indicates a neutral policy stance as no change is needed. 

    \item ``2 starting from 01.07.2016 auctions were discontinued.''\\
    \textbf{Irrelevant}: This sentence mentions that auctions were discontinued it does not provide any context related to monetary policy or a monetary policy stance.
\end{itemize}

\newpage

\begin{longtable}{p{0.118\textwidth}p{0.183\textwidth}p{0.183\textwidth}p{0.183\textwidth}p{0.183\textwidth}}
\caption{\mptitle{Central Bank of Russia}} \label{tb:cbr_mp_stance_guide}\\
\toprule
\textbf{Category} & \textbf{Hawkish} & \textbf{Dovish} & \textbf{Neutral} & \textbf{Irrelevant} \\
\midrule
\endfirsthead

\toprule
\textbf{Category} & \textbf{Hawkish} & \textbf{Dovish} & \textbf{Neutral} & \textbf{Irrelevant} \\
\midrule
\endhead
\textbf{Economic Status} & When inflation increases/rises above the 4\% target, unemployment decreases, or economic growth is projected high. & When inflation decreases/falls below the 4\% target, unemployment increases, or economic growth is projected low. & When the unemployment rate, growth, inflation, or key rate is unchanged. & Sentence is not relevant to monetary policy. \\ 
\midrule
\textbf{Ruble Value Change} & When the ruble depreciates - CBR may raise rates or intervene in the forex market to support the ruble. & When the ruble appreciates - CBR may intervene to slow appreciation or cut rates to reduce attractiveness of ruble assets. & N/A & Sentence is not relevant to monetary policy. \\ 
\midrule
\textbf{Oil/Energy Prices} & When oil/energy prices increase, when their is strong performance in the oil/energy sector, when oil development and/or exports are high. & When oil/energy prices decrease, when their is poor performance in the oil/energy sector, when oil development and/or exports are low. & N/A & Sentence is not relevant to monetary policy. \\ 
\midrule
\textbf{House Prices} & When housing prices, mortgage rates, housing demand, or rental market prices increase. & When housing prices, mortgage rates, housing demand, or rental market prices decreases. & N/A & Sentence is not relevant to monetary policy. \\ 
\midrule
\textbf{Agriculture and Food Prices} & When the performance of the agricultural sector is high (e.g high exports, large crop yield) or food prices increase. & When the performance of the agricultural sector is low (e.g low exports, low crop yield) or food prices decrease. & N/A & Sentence is not relevant to monetary policy. \\ 
\midrule
\textbf{Foreign Nations \& Sanctions} & When sanctions from other countries tighten, when trading relations worsen, increase in capital inflow, or improving alliances or counter-sanctions that directly challenge Western influence. & When sanctions from other countries ease, when trading relations improve and increase capital outflow, or seeking diplomatic resolutions or negotiations to ease sanctions pressure. & Maintaining existing geopolitical relationships without significant policy shifts and when foreign trade relations remain stable. & Sentence is not relevant to monetary policy. \\ 
\midrule
\textbf{Military and Defense} & Positive effect on GDP from increase in defense spending. & Negative effect on GDP from decreased defense spending. & N/A & Sentence is not relevant to monetary policy. \\ 
\midrule
\textbf{CBR Expectations, Actions, and Assets} & CBR expects higher inflation, tightening reserve requirements. & CBR expects lower inflation, easing reserve requirements. & Expecting balanced growth and ruble/price stability. & Sentence is not relevant to monetary policy. \\ 
\midrule
\textbf{Inflation Rates (CPI)} & CPI rises, signaling inflationary pressure. & CPI (inflation rate) falls, indicating price stability or deflationary pressures. & When CPI remains stable. & Sentence is not relevant to monetary policy. \\ 
\midrule
\textbf{Money Supply} & Tightening money supply, stricter lending policies, targeting lending risks. & Increasing money supply, easing lending conditions, supporting increased liquidity. & N/A & Sentence is not relevant to monetary policy. \\ 
\midrule
\textbf{Key Rate Fluctuations} & Key rate hikes to control inflation and stabilize the ruble. & Key rate cuts stimulate economic activity via monetary easing. & When the key rate is maintained between meetings/terms. & Sentence is not relevant to monetary policy. \\ 
\midrule
\textbf{Key Words and Phrases} & Price stability, target inflation, tightening policies like rate increases or raised reserve requirements. & Economic growth, accommodating, stimulating, supporting demand, consumption, export growth, or raise in GDP. & Use of phrases such as continued, steady, monitoring, and balanced. & Sentence is not relevant to monetary policy. \\ 
\midrule
\textbf{Labor} & When productivity decreases. & When productivity increases. & N/A & Sentence is not relevant to monetary policy. \\ 
\bottomrule
\end{longtable}