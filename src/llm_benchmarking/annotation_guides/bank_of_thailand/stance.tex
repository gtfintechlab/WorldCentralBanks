\mptext{bot}{9} Inflation, Economic Growth, Copper Prices, Peso Exchange Rate, Foreign Reserves, External Demand (Trade Partners), Key Words/Phrases, Labor Market, and Energy Prices. 
\begin{itemize}
    \item \emph{Economic Status}: A sentence pertaining to the state of the economy, relating to factors such as unemployment and inflation.
    \item \emph{Thai Baht Value Change}: A sentence pertaining to the appreciation or depreciation of the Thai Baht in the foreign exchange market.
    \item \emph{Energy/House Prices}: A sentence pertaining to fluctuations in energy sectors or housing prices, indicative of broader economic conditions.
    \item \emph{Foreign Policy}: A sentence pertaining to trade relations and economic interactions between Thailand and its international partners. If not directly related, the sentence is labeled neutral.
    \item \emph{MPC Expectations/Actions/ Assets}: A sentence that discusses the MPC's projections, actions, or asset valuations, influencing monetary policy decisions.
    \item \emph{Money Supply}: A sentence that overtly discusses changes in the money supply or shifts in the demand for loans.
    \item \emph{Key Words/Phrases}: A sentence that encapsulates specific key words or phrases which classify monetary policy stances based on their conventional usage.
    \item \emph{Labor Market}: A sentence pertaining to changes in labor market performance and productivity that may affect monetary policy.
\end{itemize}


\paragraph{Examples:}
\begin{itemize}
    \item ``The latest risk-adjusted inflation forecast for 2024 eased to 3.9 percent from 4.2 percent in the previous meeting in December.''\\ 
    \textbf{Dovish}: The sentence reports a decline in inflation forecast, which reduces the urgency for tightening of monetary policy and suggest less inflationary pressure.

    \item ``Headline inflation increased in June and July driven mainly by rising transport costs such as higher domestic petroleum prices and tricycle fare hikes.''\\ 
    \textbf{Hawkish}: This sentence highlights an increase in headline inflation, suggesting potential inflationary pressures that may prompt a tighter monetary policy stance.
    
    \item ``The highlights of the discussions on the 1st December 2011 meeting were approved by the monetary board during its regular meeting held on 15th Decemeber 2011.''\\ 
    \textbf{Irrelevant}: The sentence is simply procedural and does not provide any information about the monetary policy stance.
    
    \item ``Nonetheless, inflation forecasts for 2016-2017 stayed close to the midpoint of the target range.''\\ 
    \textbf{Neutral}: The sentence indicates that the inflation forecasts remained near the midpoint of the target range, it implies that inflation is stable but does not convey any information on need for monetary tightening or easing. 
    
    \item ``However, the moderation in growth in Q1 2014, particularly in the US, as well as tighter financial conditions and geopolitical tensions in the Middle East and Russia, could dampen the overall momentum for the rest of 2014.''\\ 
    \textbf{Dovish}: This sentence indicates some downside risks to economic growth, which hints at potential easing measures.
\end{itemize}






\newpage

\begin{longtable}{p{0.118\textwidth}p{0.183\textwidth}p{0.183\textwidth}p{0.183\textwidth}p{0.183\textwidth}}
\caption{\mptitle{Bank of Thailand}} \label{tb:bot_mp_stance_guide} \\
\toprule
\textbf{Category} & \textbf{Dovish} & \textbf{Hawkish} & \textbf{Neutral} & \textbf{Irrelevant} \\
\midrule
\endfirsthead

\toprule
\textbf{Category} & \textbf{Dovish} & \textbf{Hawkish} & \textbf{Neutral} & \textbf{Irrelevant} \\
\midrule
\endhead

\textbf{Economic Status} & 
When inflation decreases, When unemployment increases, when economic growth is projected as low. & 
When inflation increases, when unemployment decreases when economic growth is projected high when economic output is higher than potential supply/actual output when economic slack falls. & 
When unemployment rate or growth is unchanged, maintained, or sustained. & 
Sentence is not relevant to monetary policy. \\
\midrule

\textbf{Thai Baht Value Change} & 
When the Thai Baht appreciates. & 
When the Thai Baht depreciates. & 
When there is no real change in Thai Baht worth. & 
Sentence is not relevant to monetary policy. \\
\midrule

\textbf{Energy/House Prices} & 
When oil/energy sectors dwindle and prices decrease, when housing prices decrease because this can indicate a smaller consumer output, and with a dovish economy there will be a larger consumer output. & 
When oil/energy sectors are augmented/prices increase, when housing prices increase because this can indicate a large consumer output, and the economy could benefit from a hawkish policy as it will help stabilize prices. & 
When there is market uncertainty in the oil/energy sectors, when housing prices have an uncertain outlook as this means there is not clear indicators on how consumer spending is. & 
Sentence is not relevant to monetary policy. \\
\midrule

\textbf{Foreign Policy} & 
When partner economics are starting to expand, Thailand’s monetary policy will be more free as Thailand’s economic output throughput is directly proportional to its trading partners. & 
When partner economies are starting to fall or become highly volatile, Thailand should be more hawkish as the partner trading profitability is uncertain. & 
When describing specifics on an unrelated foreign nation’s economic or trade policy as this is likely to have little effect on the monetary policy control for Thailand or uncertainty in partners economies. & 
Sentence is not relevant to monetary policy. \\
\midrule

\textbf{MPC Expectations/Actions/ Assets} & 
MPC expects subpar inflation, disinflation, narrowing of spread in Thai government bonds/lower yields, low debt serviceability because in these cases, it is in best interest to allow a freer monetary policy to boost spending/market growth. & 
MPC expects high inflation, widening of spread in Thai government bonds/higher yields, high debt serviceability because in these cases, the market has some room in which the monetary policy may be more strict without a huge negative effect. & 
MPC’s expectations of future economic status remainings uncertain because there is not much indicator for which way the economic status should sway. & 
Sentence is not relevant to monetary policy. \\
\midrule

\textbf{Money Supply} & 
Money supply is low, M2 measure increases, increased demand for loans. & 
Money supply is high, increased demand for goods, low demand for loans. & 
N/A. & 
Sentence is not relevant to monetary policy. \\
\midrule

\textbf{Key Words and Phrases} & 
When the stance is "accommodative", indicating a focus on “maximum employment” and “price stability”. & 
Indicating a focus on “price stability” and “sustained growth”. & 
Use of phrases “mixed”, “moderate”, “reaffirmed. & 
Sentence is not relevant to monetary policy. \\
\midrule

\textbf{Labor Market} & 
When labor markets see improvements, Thailand’s policy is more dovish because it leads to lower inflationary pressure to support economic growth and employment. & 
When labor markets see a decline, Thailand’s policy can be more hawkish because it can increase inflationary pressures due to more limited supplies. & 
When labor markets don’t see a change in the productivity of labor markets, there isn’t a change in the monetary policy. & 
Sentence is not relevant to monetary policy. \\
\bottomrule
\end{longtable}
